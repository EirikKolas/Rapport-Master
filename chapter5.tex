% !TeX root = main.tex
%===================================== CHAP 5 =================================

\chapter{Simulation Results}\label{chap:simulation-results}

- simulation setup
- stationary target
- head on situation
- crossing situation
- overtaking situation

\section{Simulation Setup}\label{sec:simulation-setup}

The B-spline MINMPC is implemented in Python using the library presented in \cref{sec:python-implementation}. The simulation is run on a laptop with an AMD Ryzen 7 5000 Series CPU and 16 GB of RAM. The simulation parameters are summarized in \cref{tab:simulation-parameters}. 

Unless otherwise specified the OS is represented with a 3rd degree uniform B-spline of length 10 over the interval $[0, 1]$. The double integrator model is used to represent the OS, which has a maximum speed of 6 m/s. In the different simulation scenarios, the target ship heading will vary to force the situations described by the COLREGS rules 8 and 13-17.

\begin{table}[H]
    \centering
    \begin{tabular}{|c|c|c|p{7cm}|}
        \hline
        Parameter & Value & Unit & Description \\
        \hline
        \rule{0pt}{2.5ex}$k$ & 3 & - & Degree of the B-spline basis \\
        \hline
        \rule{0pt}{2.5ex}$N$ & 10 & - & Number of basis functions in the B-spline basis \\
        \hline
        \rule{0pt}{2.5ex}$v_\text{max}$ & 6 & m/s & Own ship maximum speed \\
        \hline
        \rule{0pt}{2.5ex}$d_p$ & 50 & m & Minimum distance to target on the passing side \\
        \hline
        \rule{0pt}{2.5ex}$d_o$ & 100 & m & Minimum distance to target on the opposide side \\
        \hline
    \end{tabular}
    \caption{Simulation parameters.}
    \label{tab:simulation-parameters}
\end{table}


\section{Simulation Results}\label{sec:simulation-results}

\todo[inline]{Diskuter hastighetsendring vs kursendring}

\subsection{Case 1: Stationary Targets}
\label{sec:case-1-stationary-targets}

\begin{table}
    \centering
    \begin{tabular}{|c|c|c|c|c|c|c|}
        \hline
        Case & \multicolumn{2}{c|}{TS position} & $w_{time}$ & $w_{traj}$ & $w_{acc}$ & Plot \\
        \hline
        1 & (20, 0) & $\mathbf{(-20, 100)}$ & 1 & 1 & 0 & \cref{fig:stationary-target-2} \\
        \hline
        2 & (20, 0) & $\mathbf{(-10, 200)}$ & 1 & 1 & 0 & \cref{fig:stationary-target-3} \\
        \hline
        3 & (20, 0) & $\mathbf{(10, 200)}$ & 1 & 1 & 0 & \cref{fig:stationary-target-4} \\
        \hline
        4 & (20, 0) & (-10, 200) & $\mathbf{100}$ & 1 & $1/100^2$ & \cref{fig:stationary-target-5} \\
        \hline
        5 & (20, 0) & (-10, 200) & 1 & $\mathbf{100}$ & $1/100^2$ & \cref{fig:stationary-target-6} \\
        \hline
        6 & (20, 0) & (-10, 200) & 1 & 1 & $\mathbf{1/100}$ & \cref{fig:stationary-target-7} \\
        \hline
    \end{tabular}
    \caption{Simulation cases for stationary targets. The first column indicates the case number, the second column indicates the position of the target ship, and the last three columns indicate the weights used in the optimization problem.}
    \label{tab:stationary-targets}
\end{table}


\begin{figure}
    \centering
    \includesvg[width=\textwidth,pretex=\small]{fig/stationary_obstacle/1_stationary_obstacle.svg}
    \caption{Trajectory generation around a stationary target. }
    \label{fig:stationary-target}
\end{figure}

\begin{figure}
    \centering
    \begin{subfigure}[b]{\textwidth}
        \centering
        \includesvg[width=\textwidth,pretex=\small]{fig/stationary_obstacle/2_stationary_obstacles_above_both.svg}
        \caption{Trajectory generation around two stationary targets (above both).}
        \label{fig:stationary-target-2}
    \end{subfigure}
    \hfill
    \begin{subfigure}[b]{\textwidth}
        \centering
        \includesvg[width=\textwidth,pretex=\small]{fig/stationary_obstacle/2_stationary_obstacles_between.svg}
        \caption{Trajectory generation around two stationary targets (between).}
        \label{fig:stationary-target-3}
    \end{subfigure}
    \hfill
    \begin{subfigure}[b]{\textwidth}
        \centering
        \includesvg[width=\textwidth,pretex=\small]{fig/stationary_obstacle/2_stationary_obstacles_below_both.svg}
        \caption{Trajectory generation around two stationary targets (below both).}
        \label{fig:stationary-target-4}
    \end{subfigure}
    \caption{Trajectory generation around two stationary targets in different configurations.}
    \label{fig:stationary-targets-subfigures}
\end{figure}


\begin{figure}
    \centering
    \begin{subfigure}[b]{\textwidth}
        \centering
        \includesvg[width=\textwidth,pretex=\small]{fig/stationary_obstacle/2_stationary_obstacles_high_time_ref_ratio.svg}
        \caption{Trajectory generation around two stationary targets (high time reference ratio).}
        \label{fig:stationary-target-5}
    \end{subfigure}
    \hfill
    \begin{subfigure}[b]{\textwidth}
        \centering
        \includesvg[width=\textwidth,pretex=\small]{fig/stationary_obstacle/2_stationary_obstacles_low_time_ref_ratio.svg}
        \caption{Trajectory generation around two stationary targets (low time reference ratio).}
        \label{fig:stationary-target-6}
    \end{subfigure}
    \hfill
    \begin{subfigure}[b]{\textwidth}
        \centering
        \includesvg[width=\textwidth,pretex=\small]{fig/stationary_obstacle/2_stationary_obstacles_high_acc.svg}
        \caption{Trajectory generation around two stationary targets (high acceleration).}
        \label{fig:stationary-target-7}
    \end{subfigure}
    \caption{Trajectory generation around two stationary targets in different configurations.}
    \label{fig:stationary-targets-subfigures-2}
\end{figure}


\section{Oscillations around the reference trajectory}
\label{sec:oscillations}
Using the definite integral over the squared reference error as a cost function, gives oscillations in the trajectory. This is due to the fact that the integral is less sensitive to high frequency oscillations than the sum of squared coefficients of the reference error spline. The integral puts more emphasis on the overall global shape of the trajectory, while the sum of squared coefficients puts more emphasis on the local shape. The oscillations can be damped by penalizing acceleration, but this is does not remove the oscillations completely. Further increasing the penalty on acceleration will lead to very slow convergence.
The oscillations can be removed by using the sum of squared coefficients of the reference error as a cost function.

\section{A Discussion on Conservativeness and Objectives}
\label{sec:conservativeness}
In \cref{sec:b-spline-relaxation} it was mentioned that the B-spline relaxation is conservative. This means that a solution to the relaxed problem is guaranteed to be feasible for the original problem, but not necessarily optimal. This section will discuss how this conservativeness affects the performance of the B-spline MINMPC algorithm.

As a simple example, consider the problem of minimizing the distance between the OS and a reference trajectory, while ensuring that the OS does not exceed a maximum speed:
\begin{equation}
    \begin{aligned}
        \min_{\mathbf c} \quad & \|\text{ctrl}(\mathbf p_\text{os} - \mathbf p_\text{ref})\|_\text{F}^2 \\
        \text{s.t.} \quad \mathbf p_\text{os}(0) &= \mathbf p_0, \\
                     \mathbf p_\text{os} &= \mathbf B^\top(x) \mathbf c, \\
                     \mathbf p_\text{ref} &= \mathbf B_\text{ref}^\top(x) \mathbf c_\text{ref}, \\
                     \langle p_\text{os}', p_\text{os}' \rangle &\le v_\text{max}^2,
    \end{aligned}
\end{equation}
where $\mathbf p_\text{os}$ is the position of the OS and $\mathbf p_\text{ref}$ is the reference position in their respecive B-spline bases $\mathbf B$ and $\mathbf B_\text{ref}$. The $\text{ctrl}(\cdot)$ function extracts the control points of the B-spline function and $\|\cdot\|_\text{F}^2$ is the squared Frobenius norm, given by
\begin{equation}
    \|\mathbf A\|_\text{F}^2 = \sum_{i,j} a_{ij}^2,
\end{equation}
where $\mathbf A$ is a matrix and $a_{ij}$ is the element in the $i$-th row and $j$-th column of $\mathbf A$. 
 
This optimization problem is then solved with a varying number of B-spline basis functions, $N$, and the results are shown in \cref{fig:conservativeness}. 

\begin{table}[H] % Using [H] requires the float package
    \centering
    \caption{Relationship between Number of Basis Functions (N) and Degree (k)} 
    \label{tab:conservativeness_data} 
    \begin{tabular}{|c|cccc|} 
        \hline
        \textbf{N $\backslash$ k} & \textbf{2} & \textbf{3} & \textbf{4} & \textbf{5} \\ 
        \hline
        3  & 0.2000 &        &        &        \\
        4  & 0.1000 & 0.1429 &        &        \\
        5  & 0.0764 & 0.0822 & 0.1111 &        \\
        6  & 0.0758 & 0.0801 & 0.0847 & 0.0909 \\
        7  & 0.0730 & 0.0738 & 0.0765 & 0.0831 \\
        8  & 0.0690 & 0.0669 & 0.0669 & 0.0707 \\
        9  & 0.0647 & 0.0636 & 0.0656 & 0.0668 \\
        10 & 0.0624 & 0.0640 & 0.0657 & 0.0667 \\
        11 & 0.0622 & 0.0638 & 0.0646 & 0.0651 \\
        12 & 0.0620 & 0.0630 & 0.0629 & 0.0633 \\
        13 & 0.0616 & 0.0618 & 0.0622 & 0.0629 \\
        14 & 0.0610 & 0.0612 & 0.0619 & 0.0627 \\
        15 & 0.0608 & 0.0613 & 0.0619 & 0.0624 \\
        16 & 0.0608 & 0.0612 & 0.0616 & 0.0618 \\
        17 & 0.0607 & 0.0609 & 0.0611 & 0.0615 \\
        18 & 0.0606 & 0.0605 & 0.0609 & 0.0614 \\
        19 & 0.0604 & 0.0605 & 0.0609 & 0.0613 \\
        20 & 0.0604 & 0.0605 & 0.0608 & 0.0610 \\
        21 & 0.0604 & 0.0605 & 0.0606 & 0.0608 \\
        22 & 0.0604 & 0.0604 & 0.0604 & 0.0607 \\
        23 & 0.0604 & 0.0603 & 0.0605 & 0.0607 \\
        24 & 0.0603 & 0.0603 & 0.0605 & 0.0606 \\
        25 & 0.0603 & 0.0603 & 0.0604 & 0.0605 \\
        26 & 0.0603 & 0.0603 & 0.0603 & 0.0604 \\
        27 & 0.0603 & 0.0602 & 0.0603 & 0.0604 \\
        28 & 0.0602 & 0.0602 & 0.0603 & 0.0604 \\
        29 & 0.0602 & 0.0602 & 0.0603 & 0.0603 \\
        30 & 0.0602 & 0.0602 & 0.0603 & 0.0603 \\
        31 & 0.0602 & 0.0602 & 0.0602 & 0.0603 \\
        32 & 0.0602 & 0.0602 & 0.0602 & 0.0603 \\
        33 & 0.0602 & 0.0602 & 0.0602 & 0.0603 \\
        34 & 0.0602 & 0.0602 & 0.0602 & 0.0602 \\
        35 & 0.0602 & 0.0602 & 0.0602 & 0.0602 \\
        \hline
    \end{tabular}
\end{table}

\section{Performance Evaluation}
\label{sec:performance-evaluation}

Conservativeness vs. performance

Use a more spcialized solver. Bonmin was used. Could have used CPLEX, as the problem is quadratically constrainted with a quadratic objective, although it is not open source. 