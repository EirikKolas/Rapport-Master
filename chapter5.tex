% !TeX root = main.tex
%===================================== CHAP 5 =================================

\chapter{Simulation Results}\label{chap:simulation-results}

- simulation setup
- stationary target
- head on situation
- crossing situation
- overtaking situation

\section{Simulation Setup}\label{sec:simulation-setup}

The B-spline MINMPC is implemented in Python using the library presented in \cref{sec:python-implementation}. The simulation is run on a laptop with an AMD Ryzen 7 5000 Series CPU and 16 GB of RAM. The simulation parameters are summarized in \cref{tab:simulation-parameters}. 

Unless otherwise specified the OS is represented with a 3rd degree uniform B-spline of length 10 over the interval $[0, 1]$. The double integrator model is used to represent the OS, which has a maximum speed of 6 m/s. In the different simulation scenarios, the target ship heading will vary to force the situations described by the COLREGS rules 8 and 13-17.

\begin{table}[H]
    \centering
    \begin{tabular}{|c|c|c|p{7cm}|}
        \hline
        Parameter & Value & Unit & Description \\
        \hline
        \rule{0pt}{2.5ex}$k$ & 3 & - & Degree of the B-spline basis \\
        \hline
        \rule{0pt}{2.5ex}$N$ & 10 & - & Number of basis functions in the B-spline basis \\
        \hline
        \rule{0pt}{2.5ex}$v_\text{max}$ & 6 & m/s & Own ship maximum speed \\
        \hline
        \rule{0pt}{2.5ex}$d_p$ & 50 & m & Minimum distance to target on the passing side \\
        \hline
        \rule{0pt}{2.5ex}$d_o$ & 100 & m & Minimum distance to target on the opposide side \\
        \hline
    \end{tabular}
    \caption{Simulation parameters.}
    \label{tab:simulation-parameters}
\end{table}


\section{Simulation Results}\label{sec:simulation-results}

\todo[inline]{Diskuter hastighetsendring vs kursendring}

\subsection{Simulation 1: Stationary Target}
\label{sec:simulation-1-stationary-target}

\subsection{Performance Evaluation}
\label{sec:performance-evaluation}

Conservativeness vs. performance

Use a more spcialized solver. Bonmin was used. Could have used CPLEX, as the problem is quadratically constrainted with a quadratic objective, although it is not open source.