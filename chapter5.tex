% !TeX root = main.tex
%===================================== CHAP 5 =================================

\chapter{Simulation Results}\label{chap:simulation-results}

- simulation setup
- stationary target
- head on situation
- crossing situation
- overtaking situation

\section{Simulation Setup}\label{sec:simulation-setup}

The B-spline MINMPC is implemented in Python using the library presented in \cref{sec:python-implementation}. The simulation is run on a laptop with an AMD Ryzen 7 5000 Series CPU and 16 GB of RAM. The simulation parameters are summarized in \cref{tab:simulation-parameters}. 

Unless otherwise specified the OS is represented with a 3rd degree uniform B-spline of length 10 over the interval $[0, 1]$. The double integrator model in \cref{sec:double-integrator} is used to represent the OS, which has a maximum speed of 6 m/s. In the different simulation scenarios, the TS heading will vary to force the situations described by the COLREGS rules 8 and 13-17.

For all cases going forward, the solver is initialized with the position spline of the OS equal to the reference trajectory, and the normal vectors of the starboard and port side of the TSs are initialized to the normal vectors of the reference trajectory. All other optimization variables are set to zero.


\begin{table}[H]
    \centering
    \begin{tabular}{|c|c|c|p{7cm}|}
        \hline
        Parameter & Value & Unit & Description \\
        \hline
        \rule{0pt}{2.5ex}$k$ & 3 & - & Degree of the B-spline basis \\
        \hline
        \rule{0pt}{2.5ex}$N$ & 10 & - & Number of basis functions in the B-spline basis \\
        \hline
        \rule{0pt}{2.5ex}$v_\text{max}$ & 6 & m/s & Own ship maximum speed \\
        \hline
        \rule{0pt}{2.5ex}$d_p$ & 50 & m & Minimum distance to target on the passing side \\
        \hline
        \rule{0pt}{2.5ex}$d_o$ & 100 & m & Minimum distance to target on the opposide side \\
        \hline
    \end{tabular}
    \caption{Simulation parameters.}
    \label{tab:simulation-parameters}
\end{table}


\section{Simulation Results}\label{sec:simulation-results}

\todo[inline]{Diskuter hastighetsendring vs kursendring}

\subsection{Case 1: Stationary Targets}
\label{sec:case-1-stationary-targets}

To demonstrate the B-MINMPC's ability to find optimal trajectories in non-convex environments, a stationary TS is placed at the position $(20, 0)$ in the simulation environment. The TS is located at $(20, 0)$ with a minimum passing distance of 50 m. COLREGS are not considered in scenarios 1.x, as the TSs are stationary. 

All varying parameters are summarized in \cref{tab:stationary-targets}. In \cref{fig:stationary-target} the trajectory is shown to pass the TS on the south side. This is expected, as the center of the TS is located above the North=0 axis. The following scenarios in Case 1 are variations of the first scenario, where 1.1-1.3 show the effect of changing the position of the TS, and 1.4-1.6 show the effect of changing the weights in the optimization problem. The cost function is given by
\begin{equation}
    \begin{aligned}
        \min_{\mathbf c} \quad & w_\text{time} T \\
        + & w_\text{traj}\|\text{coeffs}(\mathbf p_\text{os} - \mathbf p_\text{ref})\|_\text{F}^2 \\
        + & w_\text{acc}\|\text{coeffs}((\mathbf p_\text{os} - \mathbf{p}_\text{ref})')\|_\text{F}^2
    \end{aligned}
\end{equation}



\begin{table}
    \centering
    \begin{tabular}{|c|c|c|c|c|c|c|}
        \hline
        Case & \multicolumn{2}{c|}{TS position} & $w_\text{time}$ & $w_\text{traj}$ & $w_\text{acc}$ & Plot \\
        \hline
        1.0 & (20, 0) & $-$ & 1 & 1 & 0 & \cref{fig:stationary-target} \\
        \hline
        1.1 & (20, 0) & $\mathbf{(-20, 100)}$ & 1 & 1 & 0 & \cref{fig:stationary-target-2} \\
        \hline
        1.2 & (20, 0) & $\mathbf{(-10, 200)}$ & 1 & 1 & 0 & \cref{fig:stationary-target-3} \\
        \hline
        1.3 & (20, 0) & $\mathbf{(10, 200)}$ & 1 & 1 & 0 & \cref{fig:stationary-target-4} \\
        \hline
        1.4 & (20, 0) & (-10, 200) & $\mathbf{100}$ & 1 & $1/100^2$ & \cref{fig:stationary-target-5} \\
        \hline
        1.5 & (20, 0) & (-10, 200) & 1 & $\mathbf{100}$ & $1/100^2$ & \cref{fig:stationary-target-6} \\
        \hline
        1.6 & (20, 0) & (-10, 200) & 1 & 1 & $\mathbf{1/100}$ & \cref{fig:stationary-target-7} \\
        \hline
    \end{tabular}
    \caption{Simulation cases for stationary targets. The first column indicates the case number, the second column indicates the position of the TS, and the last three columns indicate the weights used in the optimization problem.}
    \label{tab:stationary-targets}
\end{table}


\begin{figure}
    \centering
    \includesvg[width=\textwidth,pretex=\small]{fig/stationary_obstacle/1_stationary_obstacle.svg}
    \caption{\emph{Case 1.0}: Trajectory generation around a stationary target. }
    \label{fig:stationary-target}
\end{figure}

\begin{figure}
    \centering
    \begin{subfigure}[b]{\textwidth}
        \centering
        \includesvg[width=\textwidth,pretex=\small]{fig/stationary_obstacle/2_stationary_obstacles_above_both.svg}
        \caption{\emph{Case 1.1}: Trajectory generation around two stationary targets (above both).}
        \label{fig:stationary-target-2}
    \end{subfigure}
    \hfill
    \begin{subfigure}[b]{\textwidth}
        \centering
        \includesvg[width=\textwidth,pretex=\small]{fig/stationary_obstacle/2_stationary_obstacles_between.svg}
        \caption{\emph{Case 1.2}: Trajectory generation around two stationary targets (between).}
        \label{fig:stationary-target-3}
    \end{subfigure}
    \hfill
    \begin{subfigure}[b]{\textwidth}
        \centering
        \includesvg[width=\textwidth,pretex=\small]{fig/stationary_obstacle/2_stationary_obstacles_below_both.svg}
        \caption{Trajectory generation around two stationary targets (below both).}
        \label{fig:stationary-target-4}
    \end{subfigure}
    \caption{\emph{Case 1.3}: Trajectory generation around two stationary targets in different configurations.}
    \label{fig:stationary-targets-subfigures}
\end{figure}


\begin{figure}
    \centering
    \begin{subfigure}[b]{\textwidth}
        \centering
        \includesvg[width=\textwidth,pretex=\small]{fig/stationary_obstacle/2_stationary_obstacles_high_time_ref_ratio.svg}
        \caption{\emph{Case 1.4}: Trajectory generation around two stationary targets (high time reference ratio).}
        \label{fig:stationary-target-5}
    \end{subfigure}
    \hfill
    \begin{subfigure}[b]{\textwidth}
        \centering
        \includesvg[width=\textwidth,pretex=\small]{fig/stationary_obstacle/2_stationary_obstacles_low_time_ref_ratio.svg}
        \caption{\emph{Case 1.5}: Trajectory generation around two stationary targets (low time reference ratio).}
        \label{fig:stationary-target-6}
    \end{subfigure}
    \hfill
    \begin{subfigure}[b]{\textwidth}
        \centering
        \includesvg[width=\textwidth,pretex=\small]{fig/stationary_obstacle/2_stationary_obstacles_high_acc.svg}
        \caption{\emph{Case 1.6}: Trajectory generation around two stationary targets (high acceleration).}
        \label{fig:stationary-target-7}
    \end{subfigure}
    \caption{Trajectory generation around two stationary targets in different configurations.}
    \label{fig:stationary-targets-subfigures-2}
\end{figure}


\subsection{Case 2: Head-on}
\label{sec:case-2-head-on}
\begin{figure}
    \centering
    \includesvg[width=\textwidth,pretex=\small]{fig/scenarios/Head on scenario.svg}
    \caption{Trajectory generation in a head-on situation.}
    \label{fig:head-on}
\end{figure}

\subsection{Case 3: Crossing}
\label{sec:case-3-crossing}
\begin{figure}
    \centering
    \includesvg[width=\textwidth,pretex=\small]{fig/scenarios/Crossing scenario.svg}
    \caption{Trajectory generation in a crossing situation.}
    \label{fig:crossing}
\end{figure}


\subsection{Case 4: Overtaking}
\label{sec:case-4-overtaking}
\begin{figure}
    \centering
    \includesvg[width=\textwidth,pretex=\small]{fig/scenarios/Overtaking scenario.svg}
    \caption{Trajectory generation in an overtaking situation.}
    \label{fig:overtaking}
\end{figure}

Conservativeness vs. performance

Use a more spcialized solver. Bonmin was used. Could have used CPLEX, as the problem is quadratically constrainted with a quadratic objective, although it is not open source. 