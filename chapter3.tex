% !TeX root = main.tex
%===================================== CHAP 3 =================================

\chapter{Background Theory}\label{chap:background-theory}

\section{COLREGS}\label{sec:colregs}
\todo[inline]{Copy-paste prosjektoppgava uten å plagiere meg selv}
\section{LOS Guidance}
\todo[inline]{Copy-paste prosjektoppgava uten å plagiere meg selv}

\section{Optimal Control Theory}
\subsection{Semi-infinite Programming}\label{sec:semi-infinite-programming}

Semi-infinite programming is a class of optimization problems where the objective function is defined over a finite number of variables, while the constraints are defined over an infinite number of variables \citep{Bonnans2013-gt}. The general form of a semi-infinite programming problem is given by
\begin{equation}\label{eq:semi-infinite-programming}
    \begin{aligned}
        \min_{\mathbf x \in \mathbb R^n} \quad & f(\mathbf x) \\
        \text{s.t.} \quad & g(\mathbf x, \omega) \leq 0, \quad \forall \omega \in \Omega,
    \end{aligned}
\end{equation}
where $f : \mathbb R^n \to \mathbb R$ and $g : \mathbb R^n \times \Omega \to \mathbb R$ are continuous functions, and $\Omega$ is a compact metric space, usually a continuous interval. If $\Omega$ is a finite set, e.g. $\Omega = \{\omega_1,\ldots,\omega_n\}$, \cref{eq:semi-infinite-programming} is reduced to a standard non-linear programming problem, as the feasible set can then be written as $g_i(\cdot) \leq 0, i = 1,\ldots,n$, where $g_i(\cdot) := g(\cdot, \omega_i)$ \citep{Bonnans2013-gt}.

An optimal control problem in its continuous form can be formulated as a semi-infinite programming problem:
\begin{equation}\label{eq:optimal-control-problem}
    \begin{aligned}
        \min_{\mathbf{x}(\cdot), \mathbf{u}(\cdot)} \quad & \int_{T} J(\mathbf{x}(t), \mathbf{u}(t), t) \, dt \\
        \text{s.t.} \quad & \dot{\mathbf{x}}(t) = \mathbf f(\mathbf{x}(t), \mathbf{u}(t), t), \quad \mathbf{x}(t_0) = \mathbf{x}_0, \\
        & \mathbf g(\mathbf{x}(t), \mathbf{u}(t), t) \leq \mathbf 0, \quad \forall t \in T,
    \end{aligned}
\end{equation}
where $T = [t_0, t_f]$ is the time interval, $\mathbf{x}(t) \in \mathbb R^{n_x}$ is the state, $\mathbf{u}(t) \in \mathbb R^{n_u}$ is the control input, $J : \mathbb R^{n_x} \times \mathbb R^{n_u} \times \mathbb R \to \mathbb R$ is the cost functional, $\mathbf f : \mathbb R^{n_x} \times \mathbb R^{n_u} \times T \to \mathbb R^{n_x}$ is the state equation, and $\mathbf g : \mathbb R^{n_x} \times \mathbb R^{n_u} \times T \to \mathbb R^{n_g}$ are the algebraic path constraints.

There are two main approaches to solve semi-infinite programming problems: discretization and relaxation. Discretization methods transform the infinite-dimensional problem into a finite-dimensional one by sampling the infinite set of constraints $\Omega$ at a finite number of points $\omega_i \in \Omega, i = 1, \ldots, N$. The resulting problem can then be solved using standard optimization techniques. Relaxation methods also replace the infinite constraint set with a finite set of constraints, but instead of sampling the infinite set, they introduce auxiliary variables to approximate the infinite-dimensional problem. This is often done by writing $g(\cdot, \omega)$ and $f$ in terms of a finite-dimentional basis, such as B-splines, and then solving the resulting finite-dimensional problem \citep{Dierckx1993}. More details on these methods are given in \cref{sec:multiple-shooting,sec:b-spline-relaxation}.

\subsection{Multiple Shooting}\label{sec:multiple-shooting}
Multiple shooting discretizes the infinite set of constraints by splitting the time interval $T = [t_0,t_f]$ into a finite set of points $\{t_0, t_1, \ldots, t_N\}$. At each shooting node, the state and control are introduced as separate variables. The continuous dynamics are then integrated on each subinterval, typically using Runge-Kutta methods, to ensure continuity between successive nodes.
 
\subsection{B-Spline Relaxation}\label{sec:b-spline-relaxation}
\begin{itemize}
    \item B-spline basis functions
    \item Convex hull property
    \item Conservativeness
    \item Knot insertion
\end{itemize}
