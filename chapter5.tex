% !TeX root = main.tex
%===================================== CHAP 5 =================================

\chapter{Simulation Results}\label{chap:simulation-results}

A comprehensive suite of simulation scenarios was designed to validate the B-spline MINMPC framework in dynamic obstacle encounters. First, the simulation environment and solver setup are detailed in \cref{sec:simulation-setup}. Then, trajectory generation around a stationary target is examined to illustrate the handling of non-convex constraints. Subsequent cases address COLREGS-relevant encounters: head-on approaches, crossing situations, and overtaking maneuvers. For each scenario, generated trajectories are evaluated in terms of path smoothness, constraint satisfaction, partial COLREGS compliance, thereby demonstrating the method’s robustness across diverse navigational challenges.


\section{Simulation Setup}\label{sec:simulation-setup}

The double integrator model in \cref{sec:double-integrator} is used to represent the OS, which has a maximum speed of 6 m/s. The spline representing the time $\mathbf t(x)$ is defined in the simplest way possible, as a linear function $\mathbf t(x) = Tx$, where $T$ is the total time of the simulation. As all functions in the optimization problem involving this variable only uses its derivative $\mathbf t'(x)$, $\mathbf t(x)$ can be represented by a single scalar variable $T$ in the optimization problem. 
More complex time splines are possible, giving more flexibility to the length and shapes of e.g. the maneuver windows, but this is not necessary for the simulation scenarios presented in this chapter.
In the different simulation scenarios, the TS heading and speed will vary to force the situations described by the COLREGS rules 8 and 13-17.

Unless otherwise specified, the bases for the spline functions in the optimization problem are given in \cref{tab:sim-spline-basis}. Note that the basis for the OS position $\mathbf p_\text{OS}$ is defined on finer knot vector as the normal and offset vectors for the TS hyperplanes, $\mathbf n_{s,j}$, $\mathbf n_{p,j}$, $b_{s,j}$, and $b_{p,j}$. This is done to reduce the number of optimization variables and constraints in the optimization problem, and is justified by the fact that this constraint is only active during a small time window at the closest point of approach (CPA) to the TS. For the given bases in \cref{tab:sim-spline-basis}, the OS position is represented by 12 B-spline basis functions of degree 2, so if the the TS hyperplane variables had the same basis, this would result in 72 optimization variables and 184 constraints
\footnote{The vector valued splines $\mathbf n_{s,j}$ and $\mathbf n_{p,j}$ give 12 $\times$ 2 = 24 coefficients each, and the scalar splines $b_{s,j}$ and $b_{p,j}$ give 12 $\times$ 1 = 12 coefficients each for a total of 72 optimization variables (73 if the binary variable is counted). The number of constraints are attributed to the following: 32 basis functions in the degree 4 basis of $\mathbf p_\text{OS}\cdot\mathbf{n}_{s,j}$, 4 $\times$ 22 = 88 from the 4 points defining the TS in the basis of $\mathbf p_\text{TS}\cdot\mathbf n_j$, and another 2 $\times$ 32 for the normalization of the normal vectors.}, 
as opposed to 36 optimization variables and 98 constraints per TS with the bases in \cref{tab:sim-spline-basis}.
The optimization problem in \cref{eq:minmpc-compact} is solved using Bonmin \citep{bonmin2008}, which combines existing open source libraries such as Ipopt \citep{ipopt2006} for NLPs, and Cbc \citep{cbc2005} for MIPs, to solve MINLPs. Bonmin is accessed through the CasADi framework as described in \cref{sec:python-implementation}. 
For all cases going forward, the solver is initialized as follows:
\begin{algorithmic}
    \centering
    \State $\mathbf p_\text{OS}(t) \gets \mathbf p_\text{ref}(t)$
    \For{each TS $j$}
        \State $\mathbf n_{s,j}(t) \gets -\mathbf{\hat n}_\text{ref}(t)$
        \State $\mathbf n_{p,j}(t) \gets \mathbf{\hat n}_\text{ref}(t)$
    \EndFor
\end{algorithmic}
All other optimization variables are initialized to zero. Notice the opposite signs for $\mathbf n_{s,j}(t)$ and $\mathbf n_{p,j}(t)$.
This choice ensures that the normal vectors are initialized pointing toward the OS reference from the TS constraints, which is always a feasible guess if the OS can follow the reference trajectory. 
The specific direction of the normals is reasoned from the definition of $\mathbf{\hat n}_\text{ref}(t)$ in \cref{eq:reference-normal}.

\renewcommand{\arraystretch}{1.2}
\begin{table}[htbp]
    \centering
    \begin{tabular}{|c|c|c|p{7cm}|}
        \hline
        \textbf{Parameter} & \textbf{Value} & \textbf{Unit} & \textbf{Description} \\
        \hline
        % \rule{0pt}{2.5ex}$k$ & 3 & - & Degree of the B-spline basis \\
        % \hline
        % \rule{0pt}{2.5ex}$N$ & 10 & - & Number of basis functions in the B-spline basis \\
        % \hline
        \rule{0pt}{2.5ex}$v_\text{max}$ & 6 & m/s & Own ship maximum speed \\
        \hline
        \rule{0pt}{2.5ex}$d_p$ & 50 & m & Minimum distance to target on the passing side \\
        \hline
        \rule{0pt}{2.5ex}$d_o$ & 100 & m & Minimum distance to target on the opposide side \\
        \hline
        \rule{0pt}{2.5ex}$w_\text{ref}$ & - & - & Weighting coefficients for the reference error in the cost function, see \cref{eq:minmpc-compact} \\
        \hline
        \rule{0pt}{2.5ex}$w_\text{mv}$ & - & - & Weighting coefficients for the control effort in the cost function, see \cref{eq:minmpc-compact} \\
        \hline
        \rule{0pt}{2.5ex}$w_\text{acc}$ & - & - & Weighting coefficient for the acceleration in the cost function, see \cref{eq:minmpc-compact} \\
        \rule{0pt}{2.5ex}$w_\text{time}$ & - & - & Weighting coefficient for the time in the cost function, see \cref{eq:minmpc-compact} \\
        \hline
    \end{tabular}
    \caption{Simulation parameters.}
    \label{tab:simulation-parameters}
\end{table}
\renewcommand{\arraystretch}{1.0}

\renewcommand{\arraystretch}{1.2}
\begin{table}[htbp]
    \centering
    \begin{tabular}{|p{2.5cm}||c|c|l|}
        \hline
            \rule{0pt}{2.5ex}
            \textbf{Spline Variable} & \multicolumn{3}{c|}{\textbf{B-spline Basis} $\mathbf{B}_{p, \mathbf t} = [B_{i, p, \mathbf t}(x)]_{i=0}^{N-1}$} \\[0.4ex]
            \hline
            & $N$ & $p$ & $\mathbf{t}$ \\
            \hline
            \hline
            \rule{0pt}{2.5ex}
            $\mathbf{p}_\text{OS}$
            & 12 & 2 & $\mathbf u(2,12)=\left\{0, 0, 0, \frac{1}{10}, \frac{2}{10}, \ldots, \frac{8}{10}, \frac{9}{10}, 1, 1, 1\right\}$ \\[1ex]
            \hline
            \parbox{2.5cm}{%
                $\mathbf{n}_{s, j}$, $\mathbf{n}_{p, j}$, \\
                $b_{s,j}$, $b_{p,j}$
            }
            & 6 & 1 & \rule{0pt}{3.5ex}$\mathbf u(6, 1) = \left\{0, 0, \frac{2}{10}, \frac{4}{10}, \frac{6}{10}, \frac{8}{10}, 1, 1\right\}$ \\[1ex]
            \hline
            $\mathbf{p}_\text{ref}$, $\mathbf{p}_\text{TS}$, & 2 & 1 & $\{0, 0, 1, 1\}$ \\
            \hline
            \rule{0pt}{2.5ex}
            $w_\text{ref}$, $w_\text{mv}$, $w_\mathrm{acc}$ & 10 & 0 & $\left\{0, \frac{1}{10}, \frac{2}{10}, \ldots, \frac{8}{10}, \frac{9}{10}, 1\right\}$ \\[1ex]
            \hline
            $\mathbf t'=T$ & 1 & 0 & $\{0, 1\}$ (or equivalently a constant) \\
            \hline
    \end{tabular}
    \caption{Spline functions and their B-spline bases used for the simulations.}\label{tab:sim-spline-basis}
\end{table}
\renewcommand{\arraystretch}{1.0}


\section{Simulation Results}\label{sec:simulation-results}

\subsection{Stationary Targets}
\label{sec:case-1-stationary-targets}

To demonstrate the B-MINMPC's ability to find optimal trajectories in non-convex environments, a stationary TS is placed at the position $(10, 0)$ in the simulation environment. The domain of the TS is set to 50m, and the OS is constrained to stay at least 10m away from this domain. COLREGS are not considered here, as only the qualitative behavior of the B-MINMPC is studied in cases 1.x. Both the Port and starboard maneuver directions are therefore considered as passing sides. The following cases 1.1-1.6 are variations of the first scenario, where 1.1-1.3 demonstrate the effect of changing the position of the TS. In the cases 1.4-1.6 the cost weight parameters are tuned to explore how maneuverside decisions are affected by the cost function. 
All varying parameters for the scenarios are listed in \cref{tab:stationary-targets}.


\begin{table}
    \centering
    \begin{tabular}{|c|c|c|c|c|c|c|}
        \hline
        Case & \multicolumn{2}{c|}{TS position} & $w_\text{time}$ & $w_\text{ref}$ & $w_\text{mv}$ & Plot \\
        \hline
        1.0 & (10, 0) & $-$ & 1 & 1 & 0 & \cref{fig:stationary-target} \\
        \hline
        1.1 & (10, 0) & $\mathbf{(-50, 0)}$ & 1 & 1 & 0 & \cref{fig:stationary-target-2} \\
        \hline
        1.2 & (10, 0) & $\mathbf{(-10, 200)}$ & 1 & 1 & 0 & \cref{fig:stationary-target-3} \\
        \hline
        1.3 & (10, 0) & $\mathbf{(10, 200)}$ & 1 & 1 & 0 & \cref{fig:stationary-target-4} \\
        \hline
        1.4 & (10, 0) & (-10, 200) & $\mathbf{100}$ & 1 & $1/100^2$ & \cref{fig:stationary-target-5} \\
        \hline
        1.5 & (10, 0) & (-10, 200) & 1 & $\mathbf{100}$ & $1/100^2$ & \cref{fig:stationary-target-6} \\
        \hline
        1.6 & (10, 0) & (-10, 200) & 1 & 1 & $\mathbf{1/100}$ & \cref{fig:stationary-target-7} \\
        \hline
    \end{tabular}
    \caption{Simulation cases for stationary targets. The first column indicates the case number, the second column indicates the position of the TS, and the subsequent three columns indicate the weights used in the optimization problem.}
    \label{tab:stationary-targets}
\end{table}


\begin{figure}
    \centering
    \includesvg[width=\textwidth,pretex=\footnotesize]{fig/stationary_obstacle/1_stationary_obstacle.svg}
    \caption{\emph{Case 1.0}: Trajectory generation around a stationary target. }
    \label{fig:stationary-target}
\end{figure}

In Case 1.0 a solution is found that passes the TS on the south side. This is expected, as the center of the TS is located above the North=0 axis. Using the cost function based on the squared coefficients of the reference error and the trajectory is free of oscillations (The other case with the definite integral cost is not shown). This is shown in \cref{fig:stationary-target} with the blue trajectory. The colored arrows indicate the direction of the trajectory and serve as time-stamps  spaced 60 seconds apart. These are present in all North-East plots in this chapter, and in cases where there are multiple trajectories, the same color represents the same time stamp. 


\begin{figure}[hbtp]
    \centering
    \begin{subfigure}[b]{\textwidth}
        \centering
        \includesvg[width=\textwidth,pretex=\footnotesize]{fig/stationary_obstacle/2_stationary_obstacles_above_both.svg}
        \caption{\emph{Case 1.1}: Trajectory generation around two stationary targets (above both).}
        \label{fig:stationary-target-2}
    \end{subfigure}
    \hfill
    \begin{subfigure}[b]{\textwidth}
        \centering
        \includesvg[width=\textwidth,pretex=\footnotesize]{fig/stationary_obstacle/2_stationary_obstacles_between.svg}
        \caption{\emph{Case 1.2}: Trajectory generation around two stationary targets (between).}
        \label{fig:stationary-target-3}
    \end{subfigure}
    \hfill
    \begin{subfigure}[b]{\textwidth}
        \centering
        \includesvg[width=\textwidth,pretex=\footnotesize]{fig/stationary_obstacle/2_stationary_obstacles_below_both.svg}
        \caption{Trajectory generation around two stationary targets (below both).}
        \label{fig:stationary-target-4}
    \end{subfigure}
    \caption{\emph{Case 1.3}: Trajectory generation around two stationary targets in different configurations.}
    \label{fig:stationary-targets-subfigures}
\end{figure}


Cases 1.1-1.3 in \cref{fig:stationary-targets-subfigures} introduces a second TS. Here it is demonstrated that for the same initial conditions and parameters, the B-MINMPC finds different solutions regarding starboard and port passing depending on the position of the second TS. This is the main goal of this constraint formulation in this work. 

Cases 1.4-1.6 in \cref{fig:stationary-targets-subfigures-2} show the effect of changing the weights of the cost function. In \cref{fig:stationary-target-5} $w_\text{time}$ is set to 100, which means that minimizing the total time of the trajectory is prioritized. This results in a trajectory that passes both TSs with a port maneuver, as this is the fastest route. Another effect of this is that the trajectory is more aggressive, yielding sharp turns between straight segments. With $w_\text{ref}$ set to 100 in case 1.5, (\cref{fig:stationary-target-6}), the solution trajectory passes between the two TSs, as this yields the lowest overall reference error. The same argument can be made for case 1.6 from \cref{fig:stationary-target-7}, where $w_\text{mv}$ is set to $1/100$, a port side maneuver is chosen to minimize the control effort. This is a result of how the TS at $(-10, 200)$ is positioned with a bias towards the south close to the end of the reference trajectory, and so a less agressive maneuver is possible on the northern side.

In all cases, while being initialized with the reference, the B-MINMPC is able to find a solution that passes the TSs while respecting the constraints. In other words, using the mixed integer constraints presented in \cref{sec:collision-constraints} and a simple initialization scheme, a convex optimizer is able to more robustly find the optimal solution in a non-convex environment than a convex optimizer without these constraints. This is an improvement over the previous work in \cite{Thyri2022-MPC,prosjektoppgave} where a carefully crafted initial guess had to be construced to facilitate a maneuver towards the desired passing side.

\begin{figure}
    \centering
    \begin{subfigure}[b]{\textwidth}
        \centering
        \includesvg[width=\textwidth,pretex=\footnotesize]{fig/stationary_obstacle/2_stationary_obstacles_high_time_ref_ratio.svg}
        \caption{\emph{Case 1.4}: Trajectory generation around two stationary targets (high time reference ratio).}
        \label{fig:stationary-target-5}
    \end{subfigure}
    \hfill
    \begin{subfigure}[b]{\textwidth}
        \centering
        \includesvg[width=\textwidth,pretex=\footnotesize]{fig/stationary_obstacle/2_stationary_obstacles_low_time_ref_ratio.svg}
        \caption{\emph{Case 1.5}: Trajectory generation around two stationary targets (low time reference ratio).}
        \label{fig:stationary-target-6}
    \end{subfigure}
    \hfill
    \begin{subfigure}[b]{\textwidth}
        \centering
        \includesvg[width=\textwidth,pretex=\footnotesize]{fig/stationary_obstacle/2_stationary_obstacles_high_acc.svg}
        \caption{\emph{Case 1.6}: Trajectory generation around two stationary targets (high acceleration).}
        \label{fig:stationary-target-7}
    \end{subfigure}
    \caption{Trajectory generation around two stationary targets in different configurations.}
    \label{fig:stationary-targets-subfigures-2}
\end{figure}


\FloatBarrier
\subsection{Head-on, Crossing, and Overtaking Situations}
\label{sec:case-2-head-on}


Evaluating the compliance of autonomous vessels with the COLREGS through formal, quantifiable metrics presents substantial challenges. The COLREGS were originally formulated to guide human mariners, yielding a set of rules that invite subjective interpretation, resist quantification, and complicate systematic evaluation \citep{Woerner2018}.  For example, the customary practice of smaller craft yielding to larger intercontinental vessels often prevails even when protocols can assign right-of-way to the smaller vessel, underscoring the human-centric nature of these rules. \citet{Woerner2018} propose several compliance‐assessment techniques---such as computing an optimal vessel pose at the point at the closest point of approach (\acrshort{CPA}) and defining safety functions based on inter-vessel range and relative bearing---but these methods introduce numerous scenario-specific parameters that must be manually tuned according to human evaluators' judgments, thereby limiting their robustness and general applicability.


Assessing COLREGS compliance in motion planners via post hoc metrics inevitably introduces circularity: each metric must be tailored to the specific scenario it evaluates, and a good planner should be designed to optimize the same metric. 
As a result, metric design becomes as bespoke and complex as developing a COLREGS-compliant motion planner itself, precluding truly objective assessment.  As an example, take the simple case of a minimum distance requirement: When a distance constraint is correctly encoded in the optimization problem, the passing distance is guaranteed to hold by the constrained optimization, while if the distance requirement in the optimization problem is too small, it may lead to a dangerous situation in the real world.  If the minimum distance requirement is known through the metric, it may as well be encoded in the optimization problem, rendering the metric redundant. 
Instead of relying on such tuning-dependent measures, the intrinsic geometric and kinematic properties of the optimized trajectories are analyzed.  Given a set of metrics and requirements, the B-MINMPC can---by only tuning the parameters in \cref{tab:simulation-parameters}---be configured to reliably generate trajectories that are compliant with COLREGS. 

More concretely: \acrfull{DCPA} and \acrfull{TCPA} are common metrics used to evaluate the safety of a trajectory in a COLREGS context. The parameter $d_p$ in \cref{tab:simulation-parameters} is set to 50m, which gives a hard constraint on the minimum passing distance and hence the \acrshort{DCPA} between the OS and TS. As the method presented in this paper has no intrinisic way of determining the correct \acrshort{TCPA} and \acrshort{DCPA} for a given scenario, these and other COLREGS-oriented metrics are not included in the evaluation. 


The next cases serve to illustrate the robustness of the B-MINMPC in situations where COLREGS rules 8 and 13-17 are applicable. These are constructed such that an OS following an eastwards reference trajectory at $N=0$ is on a collision course with a TS at $(0, 0)$, 250 seconds into the simulation. For each scenario, eight additional simulations are run where the starting points and reference trajectories for the OS are shifted both north and south by an appropriate distance. The shifted scenarios will all be referred to as belonging to the same \emph{batch simulation}.
In these scenarios, the simulation time is fixed to $T=500$ seconds to precisely control the timing of maneuver and encounter windows. \Cref{tab:batch-params} summarizes the parameters for each case. The weights in \cref{tab:sim-spline-basis} are configured as follows:
\begin{subequations}
    \begin{align}
        w_\text{time} &= 0 \quad \text{(\emph{fixed simulation time})}  \\
        \coeffs(w_\text{ref}) &= [10, 0, 0, 0, 0, 0, 0, 0, 0, 10]^\top     \\
        \coeffs(w_\text{mv}) &= [1, 0, 0, 0, 1, 1, 0, 0, 0, 1]^\top      \\
        \coeffs(w_\text{acc}) &= \begin{bmatrix}0, 0, \frac{1}{10}, 0, 0, 0, 0, \frac{1}{10}, 0, 0\end{bmatrix}^\top
    \end{align}
\end{subequations}
These settings promote course corrections within the maneuver windows, as detailed in \cref{sec:colregs-objectives}.  Specifically, they define an encounter window of duration $(t_8-t_2)T=(0.8-0.2)500=300$ seconds and maneuver windows of duration $(t_3-t_2)T=(t_7-t_6)T=(0.3-0.2)500=50$ seconds. The increased weight on $w_\text{mv}$ within the central portion of the encounter window encourages the OS to maintain its reference speed, facilitating course alterations during the maneuver windows in accordance with COLREGS rule 8c). This rule states that if there is sufficient sea room, alteration of course alone may be the most effective action to avoid collision \citep{COLREGS}. The weights on the acceleration $w_\text{acc}$ are set to promote a constant speed and heading between the course alterations, further constraining where the OS can maneuver. Together, these weights divide the TS encounter into nine distinct phases:
\begin{enumerate}[itemsep=0.2ex, topsep=0.5ex, parsep=0pt]
    \item Initial reference following
    \item Evasive course alteration
    \item Lateral distancing maneuver
    \item Course alteration towards reference course
    \item Parallell transit alongside reference
    \item Course alteration towards reference
    \item Re-approach to reference trajectory
    \item Final course correction
    \item Resumtion of reference following
\end{enumerate}
 


\begin{figure}
    \centering
    \includesvg[width=\textwidth,pretex=\footnotesize]{fig/illustrations/weight_functions.svg}
    \caption{Support of the weighting functions for PXTE $w_\text{ref}$, reference speed $w_\text{mv}$, and acceleration $w_\text{acc}$ in the cost function.}
    \label{fig:weight-functions}
\end{figure}



\renewcommand{\arraystretch}{1.0}
\begin{table}[htbp]
    \centering
    \small
    \begin{tabular}{|c|r|l|l|l|l|}
        \hline
        \textbf{Case} & \multicolumn{1}{c|}{\textbf{TS course}} & \textbf{TS speed} & \textbf{OS Basis} & \multicolumn{1}{c|}{\textbf{Figures}} & \multicolumn{1}{c|}{\textbf{Description}} \\
        \hline
        2.0 & -40\degree, 20\degree, 40\degree & 4.5 m/s & $\mathbf B_{\mathbf u(2,12)}$ & \cref{fig:crossing} & Base Crossing scenario \\
        \hline
        2.1 & 40\degree & 4.5 m/s & $\mathbf{B}_{3, \mathbf u(3, 13)}$ & \Cref{fig:crossing-scenario-degree-3-metrics,fig:crossing-advanced-scenario-10-3} & Degree 3 basis \\
        \hline
        2.2 & 40\degree & 4.5 m/s & $\mathbf{B}_{2, \mathbf u(2, 22)}$ & \Cref{fig:crossing-advanced-scenario-20-2,fig:crossing-advanced-scenario-20-2-traj} & Refined degree 2 basis \\
        \hline
        2.3 & 40\degree & 4.5 m/s & $\mathbf{B}_{3, \mathbf u(3, 23)}$ & \Cref{fig:crossing-advanced-scenario-20-3,fig:crossing-advanced-scenario-20-3-traj} & Refined degree 3 basis \\
        \hline
        3.0 & -90\degree & 4.5 m/s & $\mathbf B_{\mathbf u(2,12)}$ & \cref{fig:head-on} & Head-on scenario \\
        \hline
        4.0 & 75\degree & 4.0 m/s & $\mathbf B_{\mathbf u(2,12)}$ & \cref{fig:overtaking} & Overtaking scenario \\
        \hline
    \end{tabular}
    \caption{Simulation parameters for the head-on, crossing, and overtaking cases. The function $\mathbf u(p,n)$ generates a uniform degree $p$ basis with $n$ basis functions.}\label{tab:batch-params}
\end{table}
\renewcommand{\arraystretch}{1.0}


\begin{figure}[htbp]
    \centering
\includesvg[width=\textwidth,pretex=\footnotesize]{fig/scenarios/crossing_advanced/crossing_advanced_scenario_10_2.svg}
    \caption{\emph{Case 2.0}: Trajectory generation in a crossing situation where the TS has a course, from top to bottom, of -40\degree, 20\degree, and 40\degree with respect to the North axis.}
    \label{fig:crossing}
\end{figure}



Case 2 is a crossing situation where the TS is heading west at 4.5 m/s with a heading $-90\degree$. For the sake of simplicity, the OSs starting above the $N=250$ axis are considered to be too far away to be in a situation requiring a COLREGS compliant maneuver, and so only the OSs starting south of the $N=250$ axis are considered. In a crossing situation, COLREGS rule 15 applies, which states that the OSs must pass the TS with a starboard maneuver. To encourage this behaviour the minimum passing distances for the port and starboard maneuver are enforced by setting $d_p=50$ m and $d_o=100$ m respectively. The specific distances are not important, as they are heavily dependent on the environment and the speeds and sizes of the OS and TSs, and have to be tuned specifically for each scenario. The important part is that the port side passing distance is larger than the starboard side passing distance, and that the port side passing distance is not so large that no solution to the optimization problem can be found given that the starboard side solution is deemed infeasible by the optimizer.


\begin{figure}
    \centering
    \begin{subfigure}[b]{\textwidth}
        \includesvg[width=\textwidth,pretex=\footnotesize]{fig/scenarios/metrics/crossing_advanced_scenario_10_2_course_40.svg}
        \caption{\emph{Case 2.0:} B-Spline basis as given in \cref{tab:sim-spline-basis}. Trajectory shown in \cref{fig:crossing} in the $40\degree$ course case.}
        \label{fig:crossing-metrics}
    \end{subfigure}
    \begin{subfigure}[b]{\textwidth}
        \includesvg[width=\textwidth,pretex=\footnotesize]{fig/scenarios/metrics/crossing_advanced_scenario_10_3_course_40.svg}
        \caption{\emph{Case 2.1:}  Case 2.0 with $\mathbf p_\text{OS}$ parameterized with a degree 3 basis. Trajectory shown in \cref{fig:crossing-advanced-scenario-10-3} in \cref{app:additional-simulation-plots}.}
        \label{fig:crossing-scenario-degree-3-metrics}
    \end{subfigure}
    \caption{Speed, change in course, and distance to the reference trajectory for the OSs in crossing scenario. The dotted lines represent the knots in the B-spline basis for the OS position $\mathbf p_\text{OS}$, and the colored solid lines each represent a single OS starting at the corresponding North coordinate given in the legend.}
    \label{fig:crossing-scenario-metrics}
\end{figure}


\subsubsection{Case 2.0: Crossing Scenario}
\Cref{fig:crossing} shows the optimal trajectories under the settings in Case 2.0 of the OSs and TS in a crossing situation. At $t=50$ seconds, the five OSs closest to $N=150$ start the starboard maneuver, and follows a course parallell to the reference trajectory from $t=200$ until the start of the second maneuver window at $t=300$ seconds where the OSs start to turn back towards their reference trajectories. This is further emphasized in \cref{fig:crossing-metrics}, where the speed, change in course, and distance to the reference trajectory for the OSs are plotted for the case where the TS has a course of 40\degree in \gls{NED} coordinates. What follows is an analysis of the trajectory properties for Case 2.0 and how the speed and course changes relate to COLREGS 8b) and 8c).

The change in course is concentrated at the beginning and end of each maneuver window. This behavior aligns with COLREGS 8b), which advises that any alteration of course or speed should be substantial enough to be readily apparent to another vessel, as opposed to a gradual and subtle change.  The course change happens over a period of 50 seconds, which is the shortest possible duration any speed or course change can take in this simulation. This is a result of the OS position $\mathbf p_\text{OS}$ being parameterized by a B-spline with a knot interval of 0.1 and the total simulation time being 500 seconds. A notable weakness with this sparse parameterization is that the course change is smaller for the cases where the OS doesn't have to maneuver as far off the reference trajectory to avoid the TS. 
This is true for the southernmost OSs in Case 2.0 where these parameters result in only $\approx3\degree$ of course deviation from the reference over $t=50$–100\,s for the OS starting at $N=-50\,$m, versus $\approx27\degree$ at $N=250\,$m. Refinement of the B-spline basis for $\mathbf p_\text{OS}$ (see Case 2.2) increases control-point density, enabling more localized course and speed adjustments, addressing this issue.

Under COLREGS 8b), a $3\degree$ alteration over 50\,s  may be insufficiently conspicuous. To remedy this, two obvious extensions to the cost functions are (i) imposing a temporary course reference during maneuver windows and (ii) specifying a target offset from the reference path at the encounter midpoint. Both require additional cost terms and user‐tuned parameters, increasing scenario‐specific complexity and undermining the planner’s flexibility. The current objective merely incentivizes---rather than enforces---course deviations, permitting an OS to maintain its heading when a maneuver is unnecessary (as seen for the extreme OSs in \cref{fig:crossing}). Unless maneuver necessity is known a priori, such constraints risk inducing spurious maneuvers. These refinements are deferred to future work; the present formulation already demonstrates robust trajectory generation given a suitible subset of COLREGS‐related metrics.


The speed is kept close to the reference speed of 6 m/s during the segments where the OSs are not maneuvering, but dips below the reference speed during course change.  This is undesirable in context of COLREGS 8c) if there is sufficient sea room, as the OSs should prioritize altering course alone over altering speed.  This is uniquely a result of the B-Spline parameterization. For a constant speed trajectory, the equation
\begin{equation}
    u^2(x) + v^2(x) = C
\end{equation}
must hold, where $u(x)$ and $v(x)$ are the speeds of a vessel in two orthogonal directions, and $C$ is a constant. This equation has solutions of the form
\begin{equation}
  u(x) = \sqrt{C}\,\cos\varphi(x), 
  \quad
  v(x) = \sqrt{C}\,\sin\varphi(x),
\end{equation}
where $\varphi(x)$ is an arbitrary (sufficiently smooth) function of $x$.  B-Splines are piecewise polynomials and cannot represent trancendental functions such as sines and cosines.  This explains the small dips in speed during the course change segments. During the straight segments,  
Even PH B-splines, which has a piecewise polynomial representation of the speed, cannot represent a constant speed trajectory in a non-trivial way as they are still fundamentally B-splines.  The same is also true for NURBS, which while they can represent circular and elliptical arcs perfectly, cannot do so with a constant speed parameterization due to the same reasons as above.  This is a fundamental limitation of B-splines and NURBS, and not a limitation of the B-MINMPC itself.  The speed dips during the course change segments can indirectly be mitigated by refining the B-spline basis for $\mathbf p_\text{OS}$, which under acceleration constraints gives a smaller angle between the control points, and thus a smaller speed reduction. This is explained in more detail in Case 2.2.



\begin{figure}
    \centering
    \begin{subfigure}[b]{\textwidth}
        \includesvg[width=\textwidth,pretex=\footnotesize]{fig/scenarios/metrics/crossing_advanced_scenario_20_2_course_40.svg}
        \caption{\emph{Case 2.2}: Uniform degree 2 basis with 22 basis functions for the OS position $\mathbf p_\text{OS}$. Accompanying trajectory shown in \cref{fig:crossing-advanced-scenario-20-2-traj} in \cref{app:additional-simulation-plots}.}
        \label{fig:crossing-advanced-scenario-20-2}
    \end{subfigure}
    \begin{subfigure}[b]{\textwidth}
        \includesvg[width=\textwidth,pretex=\footnotesize]{fig/scenarios/metrics/crossing_advanced_scenario_20_3_course_40.svg}
        \caption{\emph{Case 2.3}: Same scenario as Case 2.2, but with a B-spline basis of degree 3 for $\mathbf p_\text{OS}$. Accompanying trajectory shown in \cref{fig:crossing-advanced-scenario-20-3-traj} in \cref{app:additional-simulation-plots}.}
        \label{fig:crossing-advanced-scenario-20-3}
    \end{subfigure}
    \caption{Speed, change in course, and distance to the reference trajectory for the OSs in crossing scenario. The dotted lines represent the knots in the B-spline basis for the OS position $\mathbf p_\text{OS}$, and the colored solid lines each represent a single OS starting at the corresponding North coordinate given in the legend.}
    \label{fig:crossing-advanced-scenario-metrics}
\end{figure}


\subsubsection{Case 2.1: Degree 3 Basis for $\mathbf p_\text{OS}$}
As per \cref{tab:sim-spline-basis}, the basis for $\mathbf p_\text{OS}$ is degree 2, which gives a continuous function for position, speed and course, but is discontinuous at the knots for acceleration and course change (also seen in \cref{fig:crossing-metrics}). 
If such discontinuities are undesirable, increasing the B-spline basis degree enhances continuity at the expense of additional constraints. A higher basis degree necessitates a finer knot vector to maintain maneuver window locality, as continuity conditions introduce dependencies between neighboring control points. \Cref{fig:crossing-scenario-degree-3-metrics} illustrates this trade-off, showing a continuous course change $\dot\chi$ but extended maneuver duration with a degree 3 B-spline basis for $\mathbf p_\text{OS}$. 
The initial maneuver in the first window results in a more significant speed reduction, dropping to approximately 4.2 m/s. Subsequently, the OSs maintain a course parallel to the reference trajectory without reconverging. After further inspection it was found that the acceleration cost in $t\in[350, 400]$ s inhibits trajectory convergence. 
Comparing \cref{fig:crossing-metrics} and \cref{fig:crossing-scenario-degree-3-metrics} reveals that the discontinuous course change $\dot{\chi}$ in the former allows the OSs to sharply revert to $\dot{\chi}=0$ between maneuvers, where the acceleration cost is active, whereas the continuity constraints in the latter force a smoother transition. This creates conflicting objectives between minimizing acceleration and converging back to the reference trajectory, and in this case---with the given parameters---the acceleration cost is prioritized over the reference trajectory convergence. The basis is in other words too coarse to allow for a degree 3 parameterization of $\mathbf p_\text{OS}$ with these parameters.

\subsubsection{Case 2.2: Refined Degree 2 Basis for $\mathbf p_\text{OS}$}\label{sec:case-2-2-refined-degree-2-basis}
As mentioned, a finer knot vector for $\mathbf p_\text{OS}$ localizes the course change, and allows for the degree 3 parameterization to be used without the acceleration cost conflicting with the reference trajectory convergence. This is demonstrated in Case 2.2, where the knot vector for $\mathbf p_\text{OS}$ is refined to 22 knots, while the degree of the B-spline basis is kept at 2. The results are shown in \cref{fig:crossing-advanced-scenario-20-2} where the speed, course change, and distance to the reference trajectory for the OSs are plotted. The trajectory is shown in \cref{fig:crossing-advanced-scenario-20-2-traj}. 

With the refined basis, the course change exhibits piecewise continuity over 25-second intervals. As depicted in \cref{fig:crossing-advanced-scenario-20-2}, the OS originating at $N=250$ m achieves a course alteration of $27\degree$, mirroring the result in \cref{fig:crossing-metrics}. However, this change now occurs over a compressed timeframe of $t=50-75$ seconds, effectively doubling the rate of course alteration. This observation encapsulates the trend associated with refining the B-spline basis for $\mathbf p_\text{OS}$: the localization of course changes intensifies until the acceleration constraints become active.

The refined basis does not at all mitigate the speed reduction during maneuvers; the speed still decreases to approximately 4.8 m/s during course alterations. This phenomenon can be analyzed by examining the relationship between the speed and the consecutive control points of $\mathbf p_\text{OS}$, as well as the angle between these control points.

Let $P_i,P_{i+1},P_{i+2}\in\mathbb R^2$ be the three consecutive control points affecting the $i$-th spline segment on a uniform quadratic B-spline with knot interval length $h$.  Over $t\in[t_{i+1}, t_{i+2}]$, the derivative of $\mathbf p_\text{OS}(t)$ is given by
\begin{equation}
    \begin{aligned}
        \mathbf p_\text{OS}'(t) 
        &=\frac{t_{i+2}-t}{h^2}\;(P_{i+1}-P_i) 
        +\frac{t - t_{i+1}}{h^2}\;(P_{i+2}-P_{i+1})\\
        &=\frac{t_{i+2}-t}{h}\;P'_0
        +\frac{t - t_{i+1}}{h}\;P'_1,
    \end{aligned}
\end{equation}
where
\begin{equation}
P'_i = \tfrac1h (P_{i+1} - P_i),
\qquad
P'_{i+1} = \tfrac1h(P_{i+2} - P_{i+1}),
\end{equation}
denote the velocities at the control points for the derivative spline. When the maximum speed constraint on $\mathbf p_\text{OS}$ is active, $\|P'_i\|=\|P'_{i+1}\|=v_{\max}$, which simplifies the following equation:  Let $\theta\in[0,\pi]$ be the smallest angle between the three consecutive points $P_i,P_{i+1},P_{i+2}$ so that
\begin{equation}
P'_i\cdot P'_{i+1} = v_{\max}^2\cos\theta.
\end{equation}
Then one shows by differentiating $\|\mathbf p'_\text{OS}(t)\|_2^2$ that the minimal speed $v_{\min}$ occurs at $t^*=\tfrac12(t_{i+1}+t_{i+2})$ and is given by
\begin{equation}
v_{\min}
=\Bigl\|\tfrac12(P'_i+P'_{i+1})\Bigr\|
=v_{\max}\sqrt{\frac{1+\cos\theta}{2}}
=v_{\max}\cos\Bigl(\tfrac\theta2\Bigr).
\end{equation}
As the derived expressions for $t^*$ and $v_{\min}$ are independent of $h$, the knot interval, the minimum achievable speed is solely dependent on the turning angle $\theta$ and the maximum speed $v_{\max}$. Consequently, mitigating the transient deceleration necessitates a reduction in $\theta$ between successive control points. While knot refinement alone is insufficient, it facilitates closer approximation of the spline by the control points. This, in turn, allows the constraint on the maximum acceleration of the OS to effectively limit the angle between consecutive control points, thereby reducing the speed reduction during maneuvers.


\subsubsection{Case 2.3: Refined Degree 3 Basis for $\mathbf p_\text{OS}$}
The situation in Case 2.1 for a 13 control point degree 3 basis for $\mathbf p_\text{OS}$ is remedied by refining the basis to 23 control points, as shown in \cref{fig:crossing-advanced-scenario-20-3}. The course change is still restricted to the maneuver windows by the weights on PXTE, reference speed, and acceleration, but in this case the refined basis allows the continuous course change to occur entirely within these windows, and the OS thus converges properly back to the reference as intended.


\subsubsection{Cases 2 and 3: Head-on and Overtaking Situations}
Similar results are seen for the crossing and overtaking cases, where the only differences are number of OSs that have to maneuver and the distances they have to travel to avoid the TS. These scenarios are run with the same weight parameters as Case 2.0 (see \cref{tab:batch-params}), but with different TS courses and speeds. For completeness' sake the results are shown in \cref{fig:head-on} and \cref{fig:overtaking} for the head-on and overtaking cases respectively. 



\begin{figure}[htbp]
    \centering
    \begin{subfigure}[b]{\textwidth}
        \centering
        \includesvg[width=\textwidth,pretex=\footnotesize]{fig/scenarios/Head on scenario.svg}
        \caption{Axes are scaled proportionally to each other.}
        \label{fig:head-on-scenario-a}
    \end{subfigure}
    \begin{subfigure}[b]{\textwidth}
        \centering
        \includesvg[width=\textwidth,pretex=\footnotesize]{fig/scenarios/Head on scenario free scale.svg}
        \caption{North axis stretced for better visualization of maneuver timings.}
        \label{fig:head-on-scenario-b}
    \end{subfigure}
    \caption{\emph{Case 3}: Trajectory generation in a head-on situation.}
    \label{fig:head-on}
\end{figure}




\begin{figure}[htbp]
    \centering
    \begin{subfigure}[b]{\textwidth}
        \includesvg[width=\textwidth,pretex=\footnotesize]{fig/scenarios/Overtaking scenario.svg}
        \caption{Axes are scaled proportionally to each other.}
        \label{fig:overtaking-scenario-a}
    \end{subfigure}
    \begin{subfigure}[b]{\textwidth}
        \includesvg[width=\textwidth,pretex=\footnotesize]{fig/scenarios/Overtaking scenario free scale.svg}
        \caption{North axis stretced for better visualization of maneuver timings.}
        \label{fig:overtaking-scenario-b}
    \end{subfigure}
    \caption{\emph{Case 4}: Trajectory generation in an overtaking situation.}
    \label{fig:overtaking}
\end{figure}




\subsection{Confined Waters}\label{sec:confined-waters}
In cases where there isn't sufficient sea room for the OS to maneuver, COLREGS rule 8c) does not apply, and the best course of action isn't necessarily to alter course. Case 5 illustrates this scenario, where the OS is confined to following a reference trajectory in a narrow channel, while the TS is heading towards the OS at a 40\degree angle, as in the crossing cases 2.1-2.3. To facilitate a speed change for the OS, the XTE weight $w_\text{ref}$ is set to 10 across the entire parameter range, while the rest of the weights are kept the same as in cases 2-4. The support of the weight functions is shown in \cref{fig:weight-functions-confined-waters}. 

\Cref{fig:crossing-advanced-scenario-10-2-xte-weight-traj} shows the trajectories for the OSs and TSs in Case 5. The high XTE weight $w_\text{ref}$ keeps the OSs at the reference trajectory, as seen in \cref{fig:crossing-advanced-scenario-10-2-xte-weight} where the change in course and XTE is zero for all OSs for the entire simulation time. As in the other scenarios the costs on acceleration and reference speed still apply, and the OSs concentrate their maneuvering in the maneuver windows, but the XTE weight prevents them from deviating from the reference trajectory. The optimal trajectory for the OSs is thus to lower their speed to avoid collision with the TS, which is exactly what \cref{fig:crossing-advanced-scenario-10-2-xte-weight} shows. 
Thus, this example illustrates the flexibility of this optimization framework, where the cost function can intuitively be adapted to the scenario at hand with only changes to the weights. 

As with the other cases, the weights and knot vector for $\mathbf p_\text{OS}$ can be modified further to achieve the desired trajectory shape. In Case 5, the OSs speed up after the first maneuver window, and before the point of closest approach to the TS, which is not necessarily the desired behavior. This is however the expected behaviour under the current weight settings, as the reference speed weight $w_\text{mv}$ is set during this segment, and the acceleration weight $w_\text{acc}$ is not active. As the purpose of this case is to illustrate the flexibility of the B-MINMPC framework, no further modifications are made to the weights or knot vector for $\mathbf p_\text{OS}$. It is clear that the trajectory can be modified to achieve a desired speed profile by adjusting the weights and knot density.

\begin{figure}[phtb]
    \centering
    \includesvg[width=\textwidth,pretex=\footnotesize]{fig/illustrations/weight_functions_confined_waters.svg}
    \caption{\textit{Case 5:} Support of the weighting functions for PXTE $w_\text{ref}$, reference speed $w_\text{mv}$, and acceleration $w_\text{acc}$ in the cost function for Case 5. The PXTE weight is set to 10 across the entire parameter range.}
    \label{fig:weight-functions-confined-waters}
\end{figure}

\begin{figure}
    \centering
    \begin{subfigure}[b]{\textwidth}
        \centering
        \includesvg[width=\textwidth,pretex=\footnotesize]{fig/scenarios/crossing_advanced/crossing_advanced_scenario_10_2_xte_weight.svg}
        \caption{\textit{Case 5:} Trajectories for $\mathbf p_\text{OS}$ and $\mathbf p_\text{TS}$ in Case 5. Identically colored triangles represent the same relative time for the OSs and TSs. They are spaced 60 seconds apart, the first at $t=0$ s and the last at $t=480$ s}
        \label{fig:crossing-advanced-scenario-10-2-xte-weight-traj}
    \end{subfigure}
    \begin{subfigure}[b]{\textwidth}
        \centering
        \includesvg[width=\textwidth,pretex=\footnotesize]{fig/scenarios/metrics/crossing_advanced_scenario_10_2_xte_weight_course_40.svg}
        \caption{Speed, change in course, and distance to the reference trajectory for the OSs in Case 5}
        \label{fig:crossing-advanced-scenario-10-2-xte-weight}
    \end{subfigure}
    \caption{\textit{Case 5:} Trajectories and metrics for the crossing scenario with a degree 2 B-spline basis for $\mathbf p_\text{OS}$, where the PXTE weight is set to 10 across the entire parameter range.}
\end{figure}

% \section{Pure Reference Following}

\FloatBarrier
\subsection{PH B-Spline OS Model}\label{sec:ph-b-spline-os-model}

To assess the impact of a heading‐dependent vessel model, the crossing scenario of Case 2 using the Pythagorean‐hodograph (PH) Dubins formulation of \cref{sec:dubins-model} is repeated. The own‐ship trajectory is parameterized by two scalar B‐splines \(u(x)\) and \(v(x)\), each of degree 2 on a uniform knot vector with 0.1 spacing---identical to the basis used for \(\mathbf p_\text{OS}\) in the previous cases. All other simulation parameters (encounter geometry, solver settings, cost weights and minimum‐separation distances) remain unchanged.

The results, shown in \cref{fig:crossing-advanced-scenario-10-2-dubins}, reveal that two of the own‐ship solutions decelerate to zero speed during the maneuver windows. At those instants, the computed turn rate \(\dot\chi\) becomes undefined, producing spurious course profiles and indicating solver difficulty in maintaining forward motion under the mixed‐integer hyperplane constraints.

Although the PH B‐spline Dubins model offers closed‐form expressions for speed, curvature and turn rate---facilitating direct incorporation of richer compliance metrics into the MINMPC cost---the highly nonlinear and nonconvex nature of the kinematic constraints impedes reliable convergence. The polynomial–rational expressions for \(\dot\chi(x)\) interact poorly with the Big–M separation inequalities, leading to failure modes not seen under the simpler double‐integrator model. Preliminary tests of an alternative Dubins model formulation (Appendix \ref{app:tangent-half-angle}) produced similar solver instabilities. Further research is required to integrate realistic heading‐dependent vessel models into B‐spline MINMPC without sacrificing numerical robustness.



\begin{figure}
    \centering
    \begin{subfigure}[b]{\textwidth}
        \centering
        \includesvg[width=\textwidth,pretex=\footnotesize]{fig/scenarios/crossing_advanced/crossing_advanced_scenario_10_2_dubins.svg}
        \caption{\textit{Case 6:} Trajectories for the OSs and TSs in Case 2.0, where the OSs are modeled as Dubins vehicles with a maximum turning radius of 50 m. The OSs are parameterized with a degree 2 B-spline basis with 12 control points. The TS is heading west at 4.5 m/s with a course of -90\degree.}
        \label{fig:crossing-advanced-scenario-10-2-dubins-traj}
    \end{subfigure}
    \begin{subfigure}[b]{\textwidth}
        \centering
        \includesvg[width=\textwidth,pretex=\footnotesize]{fig/scenarios/metrics/crossing_advanced_scenario_10_2_dubins_course_40.svg}
        \caption{\textit{Case 6:} Speed, change in course, and distance to the reference trajectory for the OSs in Case 6. As the speeds for the OSs at $N=100$ m and $N=250$ m drops to 0 m/s, the course change is undefined, which explains the spikes in the course change metric.}
        \label{fig:crossing-advanced-scenario-10-2-dubins-metrics}
    \end{subfigure}
    \caption{\textit{Case 6:} Trajectories and metrics for the crossing scenario with a degree 2 B-spline basis for $u(x)$ and $v(x)$ in the PH B-spline-based Dubins model defined in \cref{sec:dubins-model}. The numerical solver fails to converge to the optimal solution for certain initial conditions, resulting in spurious maneuvers. The Dubins model is not able to generate COLREGS-compliant trajectories in this scenario.}
    \label{fig:crossing-advanced-scenario-10-2-dubins}
\end{figure}

\section{Computational Tractability and Solution Stability}\label{sec:stability-and-tractability}

Simulations were conducted on a laptop with an AMD Ryzen 7 5000 series CPU and 16 GB RAM. \Cref{tab:solve-times} summarizes solve times for the one-target-ship scenarios under different models and knot spacings.

\begin{table}[htbp]
    \centering
    \small
    \begin{tabular}{|l|c|c|}
    \hline
    \textbf{Model} & \textbf{Number of basis functions} & \textbf{Solve time (s)} \\
    \hline
    Double integrator       & 12  & 0.5–1.0    \\
    Double integrator       & 22  & 5–10       \\
    PH B-spline Dubins      & 12  & 5–10       \\
    \hline
    \end{tabular}
    \caption{Solve times for one-target-ship scenarios on AMD Ryzen 7/16 GB RAM.}
    \label{tab:solve-times}
\end{table}

The 0.5–1.0 s solve time for the double-integrator model with 12 basis functions is a promising result in regards real-time re‐planning, as updates need not occur faster than once per second. The Python implementation is not optimized for performance either, giving room for faster solve times from a more efficient implementation. Finer discretization (22 basis functions) increases problem size and raises solve times to 5–10 s, which approaches on not being fast enough for certain maritime scenarios. The PH B-spline Dubins model exhibits similar solve times but suffers from reduced solver reliability and convergence failures due to its highly nonlinear and bilinear constraints.

Under the double integrator model, all non‐convexity arises from the hyperplane separation constraints. Under fixed simulation time, all model constraints become quadratic; fixing the hyperplane normals to a predefined value a priori then reduces the MINLP to a mixed-integer second order cone program (MI-SOCP), for which highly efficient solvers exist. Since expected trajectory shapes and maneuver windows are known a priori, fixing normals is a viable strategy.

In its current form, the B-MINMPC problem is a mixed-integer nonlinear program (MINLP), which neither guarantees a global optimum nor bounds on solution time.  The robust convergence seen in simulations relies on warm-starting both the own-ship spline and the hyperplane normals using the expected maneuver geometry.  Without such initialization, the solver can fail to converge or settle in suboptimal local minima.  This sensitivity is particularly pronounced in the PH Dubins model, where internal bilinear variable couplings introduce even stronger non-convexities.

Future work will therefore focus on the double‐integrator model and on simplifying the hyperplane constraints to achieve a MI-SOCP formulation, which would guarantee global optimality and substantially faster solve times.

\section{Summary}

The simulation study confirms that the B-spline MINMPC framework delivers smooth, COLREGS-compliant trajectories across a variety of encounter scenarios:

\begin{itemize}
  \item \textbf{Non-convex obstacle avoidance:}  
    The Big–M mixed‐integer formulation reliably selects port or starboard passages around static and moving targets without manual initialization.
  \item \textbf{COLREGS encounters:}  
    Head-on, crossing, and overtaking cases all yield the correct right‐of‐way decisions (Rules 8, 13–17) under eight shifted start conditions, demonstrating robustness to variations in initial geometry. 
  \item \textbf{Weight functions:}  
    Time‐, reference‐error‐, maneuver‐, and acceleration‐weights enable fine‐grained control over speed and course change trade‐offs. This flexibility comes with scenario-dependent tuning: aggressive time weights straighten trajectories, high reference weights tighten path tracking, and maneuver‐window weights concentrate course changes. 
  \item \textbf{Basis fidelity:}  
    Quadratic bases (12–24 control points) capture primary maneuvers with moderate variable counts. Knot refinement and degree elevation localize course changes and reduce speed dips at the expense of larger NLPs. Matching knot spacing and degree to maneuver duration is critical for achieving desired deviation magnitudes.
  \item \textbf{Computational tractability:}
    The framework shows promise for real-time replanning, with solve times of 0.5–1.0 s for the double‐integrator model with 12 basis functions. Increasing the number of basis functions and model complexity (e.g., PH Dubins) raises solve times to 5–10 s, which may be too slow for certain scenarios. Remedies include fixing hyperplane normals to expected values, which reduces the MINLP to a MI-SOCP with guaranteed global optimality and faster solve times.
\end{itemize}

Overall, the results validate that the B-spline MINMPC achieves a practical balance between trajectory quality and computational tractability in non-convex maritime motion‐planning tasks while facilitating COLREGS compliance through intuitive parameterization. 
No post-hoc COLREGS metrics are evaluated, as the method is not explicitly designed to enforce rule compliance, and such evaluation would not reflect the algorithm's internal priorities.
However, the reliance on scenario‐specific tuning and the complexity of COLREGS compliance metrics highlight the need for further research into more generalizable, parameter‐free evaluation methods. 
The primary strength of the B-MINMPC framework lies in its capacity to generate trajectories that adhere to COLREGS through the adjustment of a limited set of parameters, as detailed in \cref{tab:simulation-parameters}. This feature simplifies the development of COLREGS-compliant motion planners by establishing a clear relationship between parameter settings and the resultant trajectory characteristics. The set of parameters used in sumilations and their corresponding behaviour is shown in \cref{tab:simulation-parameters-summary}. Other methods such as \citet{Hagen2018,cho2021colreg,Menges2024} do not rely on such a parameterization, but at the cost of having limited control over the trajectory shape. 
Future research will focus on bridging the gap between the human-interpretable COLREGS rules and clear tractable COLREGS-compliant metrics, and using these to further automate the tuning of the B-MINMPC parameters.


\begin{table}[htbp]
    \centering
    % \small
    \begin{tabular}{|l|l|p{5cm}|}
        \hline
        \textbf{Parameter} & \textbf{Coefficient Values} & \textbf{Effect} \\
        \hline
        $w_\text{ref}$ & \rule{0pt}{2.5ex}$[10, 0, 0, 0, 0, 0, 0, 0, 0, 10]^\top$ & \multirow{3}{5cm}{Localizes course changes to within maneuver windows, and convergence to the reference trajectory outside them.} \\
        $w_\text{mv}$ & $[1, 0, 0, 0, 1, 1, 0, 0, 0, 1]^\top$ &  \\
        $w_\text{acc}$ & $\begin{bmatrix}0, 0, \frac{1}{10}, 0, 0, 0, 0, \frac{1}{10}, 0, 0\end{bmatrix}^\top$ & \\[2.5ex]
        \hline
        $w_\text{ref}$ & \rule{0pt}{2.5ex}$10$ & \multirow{3}{5cm}{Enforces strict adherence to the reference trajectory, facilitating speed over course changes in maneuver windows.} \\
        $w_\text{mv}$ & $[1, 0, 0, 0, 1, 1, 0, 0, 0, 1]^\top$ & \\
        $w_\text{acc}$ & $\begin{bmatrix}0, 0, \frac{1}{10}, 0, 0, 0, 0, \frac{1}{10}, 0, 0\end{bmatrix}^\top$ & \\[2.5ex]
        \hline
    \end{tabular}
    \caption{Summary of the parameters used in the simulation study and their effects on the trajectory generation.}
    \label{tab:simulation-parameters-summary}
\end{table}
