% !TeX root = main.tex
%=================================== APPENDIX ===============================


\chapter{PH B-spline Derivation}\label{app:ph-spline-derivation}
This appendix contains the derivation of the Pythogorean Hodograph (PH) B-spline, which was introduced in \cref{sec:pythogerean-hodograph}. At the end, formulas for the curvature and turn-rate are derived, which are used in the Dubins model in \cref{sec:dubins-model}.

The following derivation is based on \cite{Albrecht2016}.
The PH B-spline is given by the parametric curve
\begin{equation}\label{eq:ph-spline-curve}
    \mathbf r(x) = [a(x), b(x)]^\top,
\end{equation}
whose hodograph satisfies the Pythagorean condition
\begin{equation}\label{eq:ph-spline-hodograph}
    \|\mathbf r'(x)\|^2 = a'(x)^2 + b'(x)^2 = \sigma(x)^2,
\end{equation}
where $\sigma(x)$ itself is a spline function. 
Introducing the complex pre‐image
\begin{equation}
    z(x) = u(x) + i\,v(x),
\end{equation}
the hodograph is defined via the square relation
\begin{equation}\label{eq:hodograph-square}
    \mathbf r'(x) = z(x)^2.
\end{equation}
Expanding the complex square yields
\begin{equation}
    (u + i v)^2 = u^2 + 2iuv - v^2 = (u^2 - v^2) + i(2uv),
\end{equation}
and equating real and imaginary parts gives
\begin{subequations}\label{eq:hodograph-derivatives}
    \begin{align}
        a'(x) &= \Re\bigl(z(x)^2\bigr) = u(x)^2 - v(x)^2, \label{eq:hodograph-derivatives-a}\\
        b'(x) &= \Im\bigl(z(x)^2\bigr) = 2\,u(x)\,v(x), \label{eq:hodograph-derivatives-b}
    \end{align}
\end{subequations}
while the magnitude of the hodograph follows as
\begin{equation}
    \sigma(x) = \|\mathbf r'(x)\| = |z(x)|^2 = u(x)^2 + v(x)^2.
\end{equation}
Finally, integrating recovers the curve coordinates:
\begin{equation}
    \begin{aligned}
        a(x) &= a_0 + \int_{x_0}^{x}\bigl(u(\tau)^2 - v(\tau)^2\bigr)\,d\tau,\\
        b(x) &= b_0 + \int_{x_0}^{x}2\,u(\tau)\,v(\tau)\,d\tau,
    \end{aligned}
\end{equation}
where $(a_0,b_0)=\mathbf r(x_0)$. $\square$


The curvature $\kappa(x)$ and turn-rate $\dot\theta(x)$ of the curve $\mathbf r(x)$ can be derived from the hodograph. The curvature is defined as the rate of change of the unit tangent vector with respect to arc-length, and the turn-rate is the rate of change of the heading angle with respect to time.
With this parameterization of the PH B-spline, the curvature $\kappa(x)$ of the curve $\mathbf r(x)$ is given by
\begin{equation}
    \kappa(x) = \frac{a'(x)b''(x) - b'(x)a''(x)}{\|\mathbf r'(x)\|^3},
\end{equation}
where $a'(x)$ and $b'(x)$ are the first derivatives of the curve, and $a''(x)$ and $b''(x)$ are the second derivatives. Using \cref{eq:hodograph-derivatives} gives
which in terms of $u(x)$ and $v(x)$ simplifies to
\begin{equation}\label{eq:curvature}
    \kappa(x) = \frac{2(uv'-vu')}{(u^2+v^2)^2}.
\end{equation}
The turn-rate $\dot\theta(x)$ is then
\begin{equation}\label{eq:turn-rate}
    \dot\theta(x) = \kappa(x) \sigma(x) = \frac{2(uv'-vu')}{u^2+v^2},
\end{equation}
while the arc-length $s(x)$ is given by
\begin{equation}
    s(x) = \int_{x_0}^{x}\sigma(\tau)\,d\tau.
\end{equation}



\chapter{B-Spline Basis Python Implementation}\label{app:b-spline-python-implementation}


\subsection{B-spline Basis Class}

\begin{algorithm}
    \caption{Cox-de Boor recursion formula for B-spline basis functions (see \cref{eq:b-spline-recurrence}) implemented as a recursive function using memoization.}
    \label{alg:cox-de-boor}
    \begin{python}
    from functools import cache
    import numpy as np
    
    def cox_de_boor(
        t: np.array[float], k: int, i: int, x: float
    ) -> float:
        """
        Evaluate the i-th B-spline basis function of degree k at x
        using the Cox-de Boor recursion formula.
    
        :param t: The knots for the basis.
        :param k: The degree of the basis.
        :param i: The i-th basis function to evaluate.
        :param x: The parameter value to evaluate the basis function at.
        :return: The value of the i-th basis function at x.
        """
    
        @cache
        def _internal(t: tuple[float], x: float) -> float:
            if len(t) == 2:
                return int(t[0] <= x < t[1])
    
            b1 = t[-2] - t[0]
            if b1 != 0:
                b1 = (x - t[0]) / b1
                b1 *= _internal(t[:-1], x)
    
            b2 = t[-1] - t[1]
            if b2 != 0:
                b2 = (t[-1] - x) / b2
                b2 *= _internal(t[1:], x)
    
            return b1 + b2
    
        return _internal(tuple(t[i:i+k+1]), x)    
    
    \end{python}
    \end{algorithm}




\begin{algorithm}
\caption{B-spline basis class with datafields for degree and knots. The class evaluates the B-spline basis using the Cox-de Boor recursion formula (\cref{alg:cox-de-boor}).}
\label{alg:b-spline-class}
\begin{python}
import numpy as np

@dataclass(frozen=True)
class BSplineBasis:
    """
    Class representing a B-spline basis.
    """

    degree: int
    knots: np.ndarray

    def __len__(self):
        return len(self.knots) - self.degree - 1

    def __call__(self, x: float) -> Array[Float]:
        """
        Evaluates the B-spline basis at a given point x.
        """
        return [
            cox_de_boor(self.knots, self.degree, i, x)
            for i in range(len(self))
        ]

\end{python}
\end{algorithm}

\begin{algorithm}
\caption{The algorithms for B-Spline basis transformation (\cref{alg:basis-transformation}) and combination (\cref{alg:common-basis}) implemented as methods of the B-spline basis class (\cref{alg:b-spline-class}).}
\label{alg:b-spline-class-methods}
\begin{python}
class BSplineBasis:

    ...

    def transform(
        self, other: Callable[[float], FloatArray]
    ) -> FloatMatrix:
        """
        Get the transformation matrix P such that
        P.T @ self = other.

        :param other: The other basis or function to find the transformation matrix for.
        """
        points = self.greville_abscissae
        A = self(points).T
        b = np.concatenate([other(t).T for t in points], axis=0)
        return np.linalg.solve(A, b)

    def combine(
        self, other: 'BSplineBasis', degree: int
    ) -> 'BSplineBasis':
        """
        Find common basis of two bases with different degrees in the specified degree.

        :param other: The other basis to combine with.
        :param degree: The degree of the resulting basis.
        :return: A new BSplineBasis object representing the combined basis.
        """
        def raise_degree(knots, degree):
            if degree == 0:
                return knots
            return np.sort(np.concatenate([
                np.repeat(np.unique(knots), degree), knots
            ]))
        # Raise the degree of the basis functions
        self_knots = raise_degree(self.knots, degree - self.degree)
        other_knots = raise_degree(other.knots, degree - other.degree)

        # Find the max multiplicity of each knot
        unique_knots = np.unique(np.concatenate([self_knots, other_knots]))
        multiplicity = np.zeros_like(unique_knots, dtype=int)
        for i, k in enumerate(unique_knots):
            multiplicity[i] = max(
                np.sum(self_knots == k),
                np.sum(other_knots == k)
            )
        # Create new knots
        knots = np.repeat(unique_knots, multiplicity)
        return BSplineBasis(degree, knots)
\end{python}
\end{algorithm}


The B-spline basis class is implemented as a dataclass with the degree and knots as datafields. It has methods from \cref{sec:b-spline-theory}. Some of which are listed below:
\begin{itemize}
    \item 
    The \texttt{\_\_call\_\_} method is implemented to evaluate the B-spline basis functions at a given point $x$ using the Cox-de Boor recursion formula. This lets the user call the B-spline basis object as a function, e.g. \texttt{basis(x)}. The B-spline basis functions are evaluated using the Cox-de Boor recursion formula from \cref{eq:b-spline-recurrence} as shown in \cref{alg:cox-de-boor}. The algorithm is implemented using memoization to avoid recomputing the same values multiple times. The memoiziation is done over the minimal set of knots needed to evaluate the B-spline basis function, which is the $k+1$ knots surrounding the knot $t_i$ where the basis function is evaluated. This minimizes the number of function evaluations needed to be stored in memory and maximizes cache hits. The memoization is implemented using the \texttt{@cache} decorator from the functools library.
    \item
    The \texttt{transform} method (\cref{alg:b-spline-class-methods}) implements \cref{alg:basis-transformation} to find the transformation matrix $\mathbf T$ in \cref{eq:transformation-matrix-solution}.
    \item
    The \texttt{combine} method (\cref{alg:b-spline-class-methods}) implements \cref{alg:common-basis} to find the common basis of two B-spline bases with different degrees. 
\end{itemize}


% \FloatBarrier
% \subsection{B-spline Class}



\chapter{Dubins Model via Tangent Half-angle Substitution}\label{app:tangent-half-angle}

This appendix contains the derivation of Dubins model using the work in \citet{mercy2017spline} as a starting point. 
Dubins model was introduced in \cref{sec:dubins-model} but is repeated here for convenience:
\begin{subequations}\label{eq:app-dubins-model}
    \begin{align}
        \dot x &= V \cos(\chi),       \label{eq:app-dubins-x} \\
        \dot y &= V \sin(\chi),       \label{eq:app-dubins-y} \\
        \dot \chi &= \omega,          \label{eq:app-dubins-chi} \\
        |\omega| &\leq \omega_{\max}, \label{eq:app-dubins-omega} \\
        0 \leq V &\leq V_{\max},      \label{eq:app-dubins-V} 
    \end{align}
\end{subequations}
where $x$ and $y$ are the position coordinates, and $\chi$ is the heading angle. The control input is the speed $V$ and turn rate $\omega$.

In order to perform a B-spline relaxation on this model, all symbols in \cref{eq:app-dubins-model} need to be polynomial functions of each other. One way to  achieve this, is to use the tangent half-angle substitution
\begin{equation}\label{eq:tangent-half-angle}
    r \defeq \tan\left(\frac{\chi}{2}\right).
\end{equation}
In this section, the $\defeq$ symbol denotes a change of variables, to avoid confusion with the $=$ symbol used for equality in constraints. 

Now $\cos(\chi)$ and $\sin(\chi)$ can be expressed in terms of $r$ as
\begin{subequations}\label{eq:cos-sin-r}
    \begin{align}
        \cos(\chi) &\defeq \frac{1 - r^2}{1 + r^2}, \label{eq:cos-r} \\
        \sin(\chi) &\defeq \frac{2r}{1 + r^2}.      \label{eq:sin-r}
    \end{align}
\end{subequations}
To find polynomial expressions for $\dot x$ and $\dot y$, the variable $\tilde v$ is intruduced to satisfy the equation
\begin{equation}\label{eq:constraint-V}
    V = \tilde v (1 + r^2).
\end{equation}
Now, \cref{eq:app-dubins-x,eq:app-dubins-y} can be written as
\begin{subequations}\label{eq:app-dubins-xy}
    \begin{align}
        \dot x &= \tilde v (1 - r^2), \label{eq:app-dubins-x-r} \\
        \dot y &= 2\tilde v r.        \label{eq:app-dubins-y-r}
    \end{align}
\end{subequations}

Up to this point, the derivation in this section has been based on the work of \citet{mercy2017spline}, which used the bicycle model. The next step is to find a polynomial expression for \cref{eq:app-dubins-chi} in terms of $r$ and $\omega$. This is done by first differentiating \cref{eq:tangent-half-angle} with respect to time
\begin{equation}\label{eq:rdot}
    \begin{aligned}
        &&\atan(r) &= \frac{\chi}{2}, \\
        &\implies& \frac{1}{1 + r^2} \dot r &= \frac{1}{2} \dot \chi, \\
        &\implies& \dot r &= \frac{1}{2} \dot \chi (1 + r^2),
    \end{aligned}
\end{equation}
before substituting \cref{eq:app-dubins-chi} into \cref{eq:rdot} to get
\begin{equation}\label{eq:rdot-omega}
    \dot r = \frac{1}{2} \omega (1 + r^2),
\end{equation}
where the dot notation denotes differentiation with respect to time.
To fulfill the constraint on $\omega$ in \cref{eq:app-dubins-omega}, \cref{eq:rdot-omega} is solved for $\omega$ to get
\begin{equation}\label{eq:omega-r}
    \omega \defeq \frac{2 \dot r}{1 + r^2}.
\end{equation}
Substituting \cref{eq:omega-r} into \cref{eq:rdot-omega} gives
\begin{equation}\label{eq:rdot-omega-r}
    \left|\frac{2 \dot r}{1 + r^2}\right| \leq \omega_{\max},
\end{equation}
which can be split into the two polynomial constraints
\begin{subequations}\label{eq:rdot-omega-r-constraints}
    \begin{align}
        2 \dot r &\leq \omega_{\max} (1 + r^2), \label{eq:rdot-omega-r-constraint1} \\
        2 \dot r &\geq -\omega_{\max} (1 + r^2). \label{eq:rdot-omega-r-constraint2}
    \end{align}
\end{subequations}
The speed constraint in \cref{eq:app-dubins-V} can similarly be expressed in terms of $r$ and $\tilde v$ as
\begin{subequations}\label{eq:V-r}
    \begin{align}
        \tilde v &\ge 0, \\
        \tilde v (1 + r^2) &\leq V_{\max}.
    \end{align}
\end{subequations}

The full model is implemented by letting $r(t)$ and $\tilde v(t)$ be spline functions on a chosen B-spline basis under the constraints in \cref{eq:rdot-omega-r-constraints,eq:V-r}, using the expressions in \cref{eq:app-dubins-xy} to calculate $\dot x$ and $\dot y$ using the algorithms in \cref{sec:b-spline-theory}. Notice that $V$, $\chi$, and $\omega$ are now entirely removed from the model, which is fully described by $r$ and $\tilde v$. The coefficients of the B-spline representation of $\dot r$ are simply linear combinations of the coefficients of $r$, as explained in \cref{sec:derivative} and the position can be found by integrating the expressions in \cref{eq:app-dubins-xy}.

To implement the inequality constraint $f(t) \le g(t)$, for two spline functions $f(t)$ and $g(t)$, the constraint is enforced by ensuring the coefficients of the B-spline representation of $f(t) - g(t)$ are non-positive, exploiting the convex hull property of B-splines. 




\chapter{Reference Cost Function Plots}\label{app:reference-cost-plots}


This appendix contains additional plots of different choices of the Hessian for the reference cost functions explored in \cref{sec:oscillations}. The eigenvalues shown are scaled by the sum of the eigenvalues.


\begin{figure}
    \centering
    \includesvg[width=\textwidth]{fig/conservativeness/eigenvectors_refined_basis_degree_2_N_9.svg}
    \caption{Eigenvectors of $\mathbf H = \mathbf T^\top \mathbf T$ where $\mathbf T$ is the transformation matrix from a uniform quadratic B-spline with 9 basis functions to a refined basis with maximally repeated knots.}
    \label{fig:refined-basis-eigenvectors-9}
\end{figure} 

\begin{figure}
    \centering
    \includesvg[width=\textwidth]{fig/conservativeness/eigenvectors_inv_gramian_degree_2_N_9.svg}
    \caption{Eigenvectors of the inverse Gramian matrix $\mathbf H = \mathbf{G}^{-1}$for a uniform quadratic B-spline with 9 basis functions. Each point in a sequence along the x-axis corresponds to a basis function.}
    \label{fig:basis-inverse-gramian-eigenvectors-9}
\end{figure}

\begin{figure}
    \centering
    \includesvg[width=\textwidth,pretex=\small]{fig/conservativeness/eigenvectors_refined_inv_gramian_degree_2_N_9.svg}
    \caption{Eigenvectors of the inverse Gramian matrix $\mathbf H = \mathbf T^\top \mathbf{\tilde G}^{-1} \mathbf T$ where $\mathbf T$ is the transformation matrix from a uniform quadratic B-spline with 9 basis functions to a refined basis with maximally repeated knots, and $\mathbf{\tilde G}$ is the Gramian matrix of the refined B-spline basis.}
    \label{fig:refined-basis-inverse-gramian-eigenvectors-9}
\end{figure}


\begin{figure}
    \centering
    \includesvg[width=\textwidth]{fig/conservativeness/eigenvectors_inv_gramian_degree_2_N_20.svg}
    \caption{Eigenvectors of the inverse Gramian matrix $\mathbf H = \mathbf{G}^{-1}$for a uniform quadratic B-spline with 20 basis functions. Each point in a sequence along the x-axis corresponds to a basis function.}
    \label{fig:basis-inverse-gramian-eigenvectors-20}
\end{figure}

\begin{figure}
    \centering
    \includesvg[width=\textwidth]{fig/conservativeness/eigenvectors_refined_inv_gramian_degree_2_N_20.svg}
    \caption{Eigenvectors of the inverse Gramian matrix $\mathbf H = \mathbf T^\top \mathbf{\tilde G}^{-1} \mathbf T$ where $\mathbf T$ is the transformation matrix from a uniform quadratic B-spline with 20 basis functions to a refined basis with maximally repeated knots, and $\mathbf{\tilde G}$ is the Gramian matrix of the refined B-spline basis. Each point in a sequence along the x-axis corresponds to a basis function.}
    \label{fig:refined-basis-inverse-gramian-eigenvectors-20}
\end{figure}




\chapter{Failed Attempts at Reference Following}\label{app:failed-reference-following}

Finding a reference-following trajectory using B-splines can be viewed as a penalized regression problem where the reference trajectory is the data to be fitted and the B-spline is the model used to fit data. In the literature, there are two common ways of fitting a B-Spline to data: 1) O'Sullivan splines \citep{OSullivan1986} or smoothing splines, which balances smoothness and fidelity to the data by penalizing the second derivative of the spline\footnote{More specifically the integral over the square of the second derivative.}, and 2) the P-Spline \citep{Eilers1996}, which penalizes the difference of coefficients directly. The second approach achieves more flexibility, as the penalty function is independent of the degree of the B-spline \citep{Eilers2021}. Variations of the first approach is used in optimal control problem literature, to penalize the acceleration of a vehicle. Examples of this are found in \cite{mercy2017spline, van2015b}. P-Splines on the other hand heve not to the author's knowledge been used in optimal control problems, but are widely used in statistics and econometrics \citep{Eilers2021}.

During the course of this work, it was found that neither approach is suitable for the task of reference following. The O'Sullivan spline approach leads to oscillations in the trajectory, as the integral over the squared second derivative is less sensitive to high frequency oscillations than low frequency oscillations. This is due to the fact that the integral puts more emphasis on the overall global shape of the trajectory, which means that high frequency oscillations around this shape are not penalized as much. This is apparent from the same eigenvalue analysis as was done in \cref{sec:oscillations}.

The P-spline approach on the other hand is better at handling high frequency oscillations, but still has problems with overshooting the reference trajectory, as the cost on the second difference of the coefficients scale poorly with the integral cost on position error. I.e. increasing the cost on the second derivative of the spline coefficients hinders the convergence to the reference before the overshooting is removed. This can be seen in \cref{fig:conservativeness-oscillation-damping}. Here the continuous cost function is used with a penalty $P$ on the second order difference of the optimization coefficients $\mathbf c$ of the B-spline function. The penalty is given by
\begin{equation}\label{eq:conservativeness-oscillation-damping}
    \begin{aligned}
        \mathbf c &= [\mathbf c_i]_{i=0}^{N-1} =
        \text{ctrlPts}\left(\mathbf{p}_\text{os}\right), \\
        \Delta \mathbf c &= [\Delta\mathbf c_i]_{i=0}^{N-2} =
        [\mathbf c_i - \mathbf c_{i-1}]_{i=1}^{N-1}, \\
        \Delta^2 \mathbf c &= [\Delta \mathbf c_i - \Delta \mathbf c_{i-1}]_{i=1}^{N-2}, \\
        P &= \lambda \|\Delta^2 \mathbf c\|_\text{F}^2,
    \end{aligned}
\end{equation}
where $\lambda$ is a parameter that controls the strength of the penalty. The complete optimization problem is then a modified version of \cref{eq:conservativeness-optimization} and is given by
\begin{equation}\label{eq:conservativeness-optimization-penalty}
    \begin{aligned}
        \min_{\mathbf c} \quad & \int_0^1 \mathbf e(x)^\top \mathbf e(x) dx + \lambda \|\Delta^2 \mathbf c\|_\text{F}^2 \\
        \text{subject to} \quad &\mathbf p_\text{os}(0) = \mathbf p_0, \\
                     &\mathbf p_\text{os} = \mathbf B^\top(x) \mathbf c, \\
                     &\mathbf p_\text{ref} = \mathbf B_\text{ref}^\top(x) \mathbf c_\text{ref}, \\
                     &\langle p_\text{os}', p_\text{os}' \rangle \le v_\text{max}^2.
    \end{aligned}
\end{equation}
where the symbols are as defined in \cref{eq:conservativeness-objective}.
For small penalties e.g. $\lambda=0.01$, the trajectory approaches that of \cref{fig:conservativeness-traj-integral}. Increasing the penalty to $\lambda=100$ gives a trajectory that only reaches about half the distance to the end of the reference. For no $\lambda$ in between does the trajectory not overshoot the North = 0m axis. As neither oscillations nor overshooting is desirable behaviour, using the integral over the squared reference error as a cost function is not a good choice. Adding the penalty term still leads to overshooting the reference. These two methods are therefore not suitible for the final implementation.

\begin{figure}
    \centering
    \includesvg[width=\textwidth]{fig/conservativeness/p_spline_degree_3_n_10.svg}
    \caption{Continuous cost with $N=10$, $p=3$ and a penalty on the second order difference of the optimization coefficients. The trajectory approaches that of \cref{fig:conservativeness-traj-integral} as $\lambda$ is decreased.}\label{fig:conservativeness-oscillation-damping}
\end{figure}


\chapter{Additional Simulation Plots}\label{app:additional-simulation-plots}

This appendix contains additional simulation plots for the scenarios in \cref{sec:simulation-results}.


\begin{figure}
    \centering
    \includesvg[width=\textwidth]{fig/scenarios/crossing_advanced/crossing_advanced_scenario_10_3.svg}
    \caption{\emph{Case 2.1}: Optimal trajectory in a crossing scenario where the TS has a course, from top to bottom, of $-40\degree$, $20\degree$, and $40\degree$ with respect to the North axis. This is a variation of \emph{Case 2.0} shown in \cref{fig:crossing}, but where the position of the OS $\mathbf p_\text{os}$ is represented in a degree 3 uniform basis with 13 basis functions.}
    \label{fig:crossing-advanced-scenario-10-3}
\end{figure}

\begin{figure}
    \centering
    \begin{subfigure}[b]{\textwidth}
        \centering
        \includesvg[width=\textwidth]{fig/scenarios/crossing_advanced/crossing_advanced_scenario_20_2.svg}
        \caption{\emph{Case 2.2}: Optimal trajectory in a crossing scenario where the TS has a course, from top to bottom, of $-40\degree$, $20\degree$, and $40\degree$ with respect to the North axis. This is a variation of \emph{Case 2.0} shown in \cref{fig:crossing}, but where the position of the OS $\mathbf p_\text{os}$ is represented in a degree 2 uniform basis with 21 basis functions.}
        \label{fig:crossing-advanced-scenario-20-2-traj}        
    \end{subfigure}
    \begin{subfigure}[b]{\textwidth}
        \centering
        \includesvg[width=\textwidth]{fig/scenarios/crossing_advanced/crossing_advanced_scenario_20_3.svg}
        \caption{\emph{Case 2.3}: Optimal trajectory in a crossing scenario where the TS has a course, from top to bottom, of $-40\degree$, $20\degree$, and $40\degree$ with respect to the North axis. This is a variation of \emph{Case 2.0} shown in \cref{fig:crossing}, but where the position of the OS $\mathbf p_\text{os}$ is represented in a degree 3 uniform basis with 21 basis functions.}
        \label{fig:crossing-advanced-scenario-20-3-traj}
    \end{subfigure}
    \caption{Optimal trajectories in a crossing scenario with different B-spline basis representations of the OS position $\mathbf p_\text{os}$. The TS has a course, from top to bottom, of $-40\degree$, $20\degree$, and $40\degree$ with respect to the North axis.}
\end{figure}