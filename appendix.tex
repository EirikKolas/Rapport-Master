% !TeX root = main.tex
%=================================== APPENDIX ===============================


\chapter{Homogeneous Integration Matrix}\label{app:homogeneous-integration-matrix}

To correctly define the integration operator $\mathbf{T}_I$ in matrix form while preserving degrees of freedom, we extend the system using homogeneous coordinates. This ensures that integration does not lose information due to the absence of a known constant term.

\section{Homogeneous Representation of Integration}

In \cref{eq:b-spline-integral-matrix}, the coefficients of the integral B-spline are expressed as
\begin{equation}
    \mathbf{c}_I = \mathbf T_I \mathbf{c}_D + \mathbf{c}_0.
\end{equation}
By prepending the coefficient vector $\mathbf{c}_D$ with 1, we can write this whole expression in homogeneous form as
\begin{equation}
    \mathbf{c}_I = 
    \underbrace{
        \begin{bmatrix}
            \mathbf{c}_0 & \mathbf T_I
        \end{bmatrix} 
    } _{\mathbf{\tilde T}_I}
    \underbrace{
        \begin{bmatrix}
            1 \\
            \mathbf{c}_D
        \end{bmatrix}
    }_{\mathbf{\tilde c}_D}.
\end{equation}

By inverting the transformation matrix $\mathbf{\tilde T}_I$, we can express the derivative coefficients as
\begin{equation}
    \mathbf{\tilde c}_D = \mathbf{\tilde T}_I^{-1} \mathbf{c}_I.
\end{equation}

Proof:


\chapter{B-Spline Basis Python Implementation}\label{app:b-spline-python-implementation}


\subsection{B-spline Basis Class}

\begin{algorithm}
    \caption{Cox-de Boor recursion formula for B-spline basis functions (see \cref{eq:b-spline-recurrence}) implemented as a recursive function using memoization.}
    \label{alg:cox-de-boor}
    \begin{python}
    from functools import cache
    import numpy as np
    
    def cox_de_boor(
        t: np.array[float], k: int, i: int, x: float
    ) -> float:
        """
        Evaluate the i-th B-spline basis function of degree k at x
        using the Cox-de Boor recursion formula.
    
        :param t: The knots for the basis.
        :param k: The degree of the basis.
        :param i: The i-th basis function to evaluate.
        :param x: The parameter value to evaluate the basis function at.
        :return: The value of the i-th basis function at x.
        """
    
        @cache
        def _internal(t: tuple[float], x: float) -> float:
            if len(t) == 2:
                return int(t[0] <= x < t[1])
    
            b1 = t[-2] - t[0]
            if b1 != 0:
                b1 = (x - t[0]) / b1
                b1 *= _internal(t[:-1], x)
    
            b2 = t[-1] - t[1]
            if b2 != 0:
                b2 = (t[-1] - x) / b2
                b2 *= _internal(t[1:], x)
    
            return b1 + b2
    
        return _internal(tuple(t[i:i+k+1]), x)    
    
    \end{python}
    \end{algorithm}




\begin{algorithm}
\caption{B-spline basis class with datafields for degree and knots. The class evaluates the B-spline basis using the Cox-de Boor recursion formula (\cref{alg:cox-de-boor}).}
\label{alg:b-spline-class}
\begin{python}
import numpy as np

@dataclass(frozen=True)
class BSplineBasis:
    """
    Class representing a B-spline basis.
    """

    degree: int
    knots: np.ndarray

    def __len__(self):
        return len(self.knots) - self.degree - 1

    def __call__(self, x: float) -> Array[Float]:
        """
        Evaluates the B-spline basis at a given point x.
        """
        return [
            cox_de_boor(self.knots, self.degree, i, x)
            for i in range(len(self))
        ]

\end{python}
\end{algorithm}

\begin{algorithm}
\caption{The algorithms for B-Spline basis transformation (\cref{alg:basis-transformation}) and combination (\cref{alg:common-basis}) implemented as methods of the B-spline basis class (\cref{alg:b-spline-class}).}
\label{alg:b-spline-class-methods}
\begin{python}
class BSplineBasis:

    ...

    def transform(
        self, other: Callable[[float], FloatArray]
    ) -> FloatMatrix:
        """
        Get the transformation matrix P such that
        P.T @ self = other.

        :param other: The other basis or function to find the transformation matrix for.
        """
        points = self.greville_abscissae
        A = self(points).T
        b = np.concatenate([other(t).T for t in points], axis=0)
        return np.linalg.solve(A, b)

    def combine(
        self, other: 'BSplineBasis', degree: int
    ) -> 'BSplineBasis':
        """
        Find common basis of two bases with different degrees in the specified degree.

        :param other: The other basis to combine with.
        :param degree: The degree of the resulting basis.
        :return: A new BSplineBasis object representing the combined basis.
        """
        def raise_degree(knots, degree):
            if degree == 0:
                return knots
            return np.sort(np.concatenate([
                np.repeat(np.unique(knots), degree), knots
            ]))
        # Raise the degree of the basis functions
        self_knots = raise_degree(self.knots, degree - self.degree)
        other_knots = raise_degree(other.knots, degree - other.degree)

        # Find the max multiplicity of each knot
        unique_knots = np.unique(np.concatenate([self_knots, other_knots]))
        multiplicity = np.zeros_like(unique_knots, dtype=int)
        for i, k in enumerate(unique_knots):
            multiplicity[i] = max(
                np.sum(self_knots == k),
                np.sum(other_knots == k)
            )
        # Create new knots
        knots = np.repeat(unique_knots, multiplicity)
        return BSplineBasis(degree, knots)
\end{python}
\end{algorithm}


The B-spline basis class is implemented as a dataclass with the degree and knots as datafields. It has methods from \cref{chap:b-spline-theory}. Some of which are listed below:
\begin{itemize}
    \item 
    The \texttt{\_\_call\_\_} method is implemented to evaluate the B-spline basis functions at a given point $x$ using the Cox-de Boor recursion formula. This lets the user call the B-spline basis object as a function, e.g. \texttt{basis(x)}. The B-spline basis functions are evaluated using the Cox-de Boor recursion formula from \cref{eq:b-spline-recurrence} as shown in \cref{alg:cox-de-boor}. The algorithm is implemented using memoization to avoid recomputing the same values multiple times. The memoiziation is done over the minimal set of knots needed to evaluate the B-spline basis function, which is the $k+1$ knots surrounding the knot $t_i$ where the basis function is evaluated. This minimizes the number of function evaluations needed to be stored in memory and maximizes cache hits. The memoization is implemented using the \texttt{@cache} decorator from the functools library.
    \item
    The \texttt{transform} method (\cref{alg:b-spline-class-methods}) implements \cref{alg:basis-transformation} to find the transformation matrix $\mathbf T$ in \cref{eq:transformation-matrix-solution}.
    \item
    The \texttt{combine} method (\cref{alg:b-spline-class-methods}) implements \cref{alg:common-basis} to find the common basis of two B-spline bases with different degrees. 
\end{itemize}


\FloatBarrier
\subsection{B-spline Class}



\chapter{Dubins Model via Tangent Half-angle Substitution}\label{app:tangent-half-angle}

This appendix contains the derivation of Dubins model using the work in \citet{mercy2017spline} as a starting point. 
Dubins model was introduced in \cref{sec:dubins-model} but is repeated here for convenience:
\begin{subequations}\label{eq:app-dubins-model}
    \begin{align}
        \dot x &= V \cos(\chi),       \label{eq:app-dubins-x} \\
        \dot y &= V \sin(\chi),       \label{eq:app-dubins-y} \\
        \dot \chi &= \omega,          \label{eq:app-dubins-chi} \\
        |\omega| &\leq \omega_{\max}, \label{eq:app-dubins-omega} \\
        0 \leq V &\leq V_{\max},      \label{eq:app-dubins-V} 
    \end{align}
\end{subequations}
where $x$ and $y$ are the position coordinates, and $\chi$ is the heading angle. The control input is the speed $V$ and turn rate $\omega$.

In order to perform a B-spline relaxation on this model, all symbols in \cref{eq:app-dubins-model} need to be polynomial functions of each other. One way to  achieve this, is to use the tangent half-angle substitution
\begin{equation}\label{eq:tangent-half-angle}
    r \defeq \tan\left(\frac{\chi}{2}\right).
\end{equation}
In this section, the $\defeq$ symbol denotes a change of variables, to avoid confusion with the $=$ symbol used for equality in constraints. 

Now $\cos(\chi)$ and $\sin(\chi)$ can be expressed in terms of $r$ as
\begin{subequations}\label{eq:cos-sin-r}
    \begin{align}
        \cos(\chi) &\defeq \frac{1 - r^2}{1 + r^2}, \label{eq:cos-r} \\
        \sin(\chi) &\defeq \frac{2r}{1 + r^2}.      \label{eq:sin-r}
    \end{align}
\end{subequations}
To find polynomial expressions for $\dot x$ and $\dot y$, the variable $\tilde v$ is intruduced to satisfy the equation
\begin{equation}\label{eq:constraint-V}
    V = \tilde v (1 + r^2).
\end{equation}
Now, \cref{eq:app-dubins-x,eq:app-dubins-y} can be written as
\begin{subequations}\label{eq:app-dubins-xy}
    \begin{align}
        \dot x &= \tilde v (1 - r^2), \label{eq:app-dubins-x-r} \\
        \dot y &= 2\tilde v r.        \label{eq:app-dubins-y-r}
    \end{align}
\end{subequations}

Up to this point, the derivation in this section has been based on the work of \citet{mercy2017spline}, which used the bicycle model. The next step is to find a polynomial expression for \cref{eq:app-dubins-chi} in terms of $r$ and $\omega$. This is done by first differentiating \cref{eq:tangent-half-angle} with respect to time
\begin{equation}\label{eq:rdot}
    \begin{aligned}
        &&\atan(r) &= \frac{\chi}{2}, \\
        &\implies& \frac{1}{1 + r^2} \dot r &= \frac{1}{2} \dot \chi, \\
        &\implies& \dot r &= \frac{1}{2} \dot \chi (1 + r^2),
    \end{aligned}
\end{equation}
before substituting \cref{eq:app-dubins-chi} into \cref{eq:rdot} to get
\begin{equation}\label{eq:rdot-omega}
    \dot r = \frac{1}{2} \omega (1 + r^2),
\end{equation}
where the dot notation denotes differentiation with respect to time.
To fulfill the constraint on $\omega$ in \cref{eq:app-dubins-omega}, \cref{eq:rdot-omega} is solved for $\omega$ to get
\begin{equation}\label{eq:omega-r}
    \omega \defeq \frac{2 \dot r}{1 + r^2}.
\end{equation}
Substituting \cref{eq:omega-r} into \cref{eq:rdot-omega} gives
\begin{equation}\label{eq:rdot-omega-r}
    \left|\frac{2 \dot r}{1 + r^2}\right| \leq \omega_{\max},
\end{equation}
which can be split into the two polynomial constraints
\begin{subequations}\label{eq:rdot-omega-r-constraints}
    \begin{align}
        2 \dot r &\leq \omega_{\max} (1 + r^2), \label{eq:rdot-omega-r-constraint1} \\
        2 \dot r &\geq -\omega_{\max} (1 + r^2). \label{eq:rdot-omega-r-constraint2}
    \end{align}
\end{subequations}
The speed constraint in \cref{eq:app-dubins-V} can similarly be expressed in terms of $r$ and $\tilde v$ as
\begin{subequations}\label{eq:V-r}
    \begin{align}
        \tilde v &\ge 0, \\
        \tilde v (1 + r^2) &\leq V_{\max}.
    \end{align}
\end{subequations}

The full model is implemented by letting $r(t)$ and $\tilde v(t)$ be spline functions on a chosen B-spline basis under the constraints in \cref{eq:rdot-omega-r-constraints,eq:V-r}, using the expressions in \cref{eq:app-dubins-xy} to calculate $\dot x$ and $\dot y$ using the algorithms in \cref{chap:b-spline-theory}. Notice that $V$, $\chi$, and $\omega$ are now entirely removed from the model, which is fully described by $r$ and $\tilde v$. The coefficients of the B-spline representation of $\dot r$ are simply linear combinations of the coefficients of $r$, as explained in \cref{sec:derivative} and the position can be found by integrating the expressions in \cref{eq:app-dubins-xy}.

To implement the inequality constraint $f(t) \le g(t)$, for two spline functions $f(t)$ and $g(t)$, the constraint is enforced by ensuring the coefficients of the B-spline representation of $f(t) - g(t)$ are non-positive, exploiting the convex hull property of B-splines. 