% !TeX root = main.tex
%===================================== CHAP 2 =================================

\chapter{B-spline Theory}
\label{chap:b-spline-theory}

The goal of this chapter is to provide enough information about B-splines to understand the subsequent chapters in addition to being able to implement B-splines in a numerical optimization context from scratch. The chapter will cover the definition of B-splines, their properties, and operations that can be performed on them.

\section{Definition and Properties}
The following definition is from \cite{Grimstad2016}:
A B-spline $f: \mathbb R \rightarrow \mathbb R$ is given by $n$ \emph{B-spline coefficients} $\mathbf c = [c_j]_{j=0}^{n-1}$ and $n+p+1$ non-decreasing knots $\mathbf t = [t_j]_{j=0}^{n+p}$ as follows:

\begin{equation}\label{eq:b-spline-def}
    f(x ; \mathbf{c}, p, \mathbf{t})=\sum_{j=0}^{n-1} c_j B_{j, p, \mathbf{t}}(x)=\mathbf{c}^{\top} \mathbf{B}_{\mathrm{p}, \mathrm{t}}(\mathrm{x})
\end{equation}

When the parameters $\mathbf{c}, p$, and $\mathbf{t}$ are given by the context, $f(x ; \mathbf{c}, p, \mathbf{t})$ is simply denoted $f(x)$. The sequence of coefficients $c_j$ control the shape of the B-spline, and the degree $p$ is a non-negative integer. The B-spline basis functions $B_{j, p, \mathbf{t}}(x)$ are defined recursively in terms of the degree $p$ and the knots $\mathbf t$ to form the column vector $\mathbf{B}_{p, \mathbf{t}}(x) = [B_{j, p, \mathbf{t}}(x)]_{j=0}^{n-1}$. 
in \cref{eq:b-spline-def}. These B-spline basis functions are defined as

\begin{subequations}\label{eq:b-spline-recurrence}
    \begin{align}
        B_{j, p, \mathbf{t}}(x) & =\frac{x-t_j}{t_{j+p}-t_j} B_{j, p-1, \mathrm{t}}(x)+\frac{t_{j+1+p}-x}{t_{j+1+p}-t_{j+1}} B_{j+1, p-1, \mathrm{t}}(x) \label{eq:b-spline-recurrence-p} \\
        B_{j, 0, \mathbf{t}}(x) & := 
        \mathbf 1_{[t_j, t_{j+1})} =
        \begin{cases}
            1, & t_j \leq x<t_{j+1} \\
            0, & \text { otherwise }
        \end{cases} \label{eq:b-spline-recurrence-0}
    \end{align}
\end{subequations}

From \cref{eq:b-spline-recurrence}, it is clear that the B-spline basis functions are all non-negative, as \cref{eq:b-spline-recurrence-0} gives $B_{j, 0, \mathbf{t}}(x) \geq 0 \quad\forall j, x$, and the coefficients $\frac{x-t_j}{t_{j+p}-t_j}$ and $\frac{t_{j+1+p}-x}{t_{j+1+p}-t_{j+1}}$
in \cref{eq:b-spline-recurrence-p} are also non-negative $\forall x, j\in[0,\dots,n-1], p, \mathbf t$ by the non-decreasing condition on the knots $\mathbf t$. 

\Cref{eq:b-spline-recurrence} also implies $B_{j, p, \mathbf{t}}(x)$ has local support on the interval $\left[t_j, t_{j+p+1}\right.)$. This can be seen by noting that the support of $B_{j, 0, \mathbf{t}}$, $\text{supp}(B_{j, 0, \mathbf{t}}) = \left[t_j, t_{j+1}\right.)$ and for each of the $p$ recurrence steps in \cref{eq:b-spline-recurrence-p}, the support is extended by one knot. 

It can also be shown that the B-spline basis functions are a partition of unity, i.e. $\sum_{j=0}^{n-1} B_{j, p, \mathbf{t}}(x) = 1 \quad\forall x \in \left[t_0, t_{n+p}\right.)$ \citep{deBoor1978practicalguide}.

These properties are well-documented in the literature and are summarized as follows:
\begin{property}[Nonnegativity]\label{b-prop:nonnegativity}
    $B_{j, p, \mathbf{t}}(x) \geq 0 \quad\forall j, p$, and $x$.
\end{property}

\begin{property}[Local support]\label{b-prop:localsupport}
    $B_{j, p, \mathbf{t}}(x)=0 \quad\forall x \notin\left[t_j, t_{j+p+1}\right.)$.
\end{property}

\begin{property}[Partition of unity]\label{b-prop:partitionofunity}
    $\sum_{j=i-p}^i B_{j, p, \mathbf{t}}(x)=1 \quad\forall x \in\left[t_i, t_{i+1}\right.)$.
\end{property}

So far, only B-splines with a one-dimensional co-domain have been considered. However, B-splines can be extended to curves in higher dimensions by simply allowing the coefficients $c_j$ to be vectors. In this case the B-spline is a vector-valued function $f: \mathbb R \rightarrow \mathbb R^m$. The \cref{b-prop:nonnegativity,b-prop:localsupport,b-prop:partitionofunity} still hold, as the basis functions themselves are still scalar-valued.

\begin{figure}
    \centering
    \includesvg[width=0.8\textwidth]{fig/b-spline/knots + control points.svg}
    \caption{A 3rd degree B-spline (full) with 4 control points, knot vector \raggedright{$\mathbf t = [0, 0, 0, 0.5, 1, 1, 1]$} and corresponding control polygon (dashed).}
    \label{fig:b-spline-knots-control-points}
\end{figure}

Together, \cref{b-prop:nonnegativity,b-prop:partitionofunity} imply that the B-spline in \cref{eq:b-spline-def} is a convex combination of the coefficients $\mathbf c$. This means that the B-spline is always within the convex hull of the control points $\mathbf c$.

\begin{property}[Convex hull]\label{b-prop:convexhull}
    $f(x) \in \text{conv}(\mathbf{c})
    \quad\forall x$.
\end{property}

The convex hull $\text{conv}(\mathbf{c})$ of a set of points $\mathbf{c} = \{c_1, c_2, \ldots, c_n\}$ is defined as the set of all convex combinations of the points. Mathematically, it is given by:

\begin{equation}
    \text{conv}(\mathbf{c}) = \left\{ \sum_{i=1}^n \lambda_i c_i \mid \lambda_i \geq 0, \sum_{i=1}^n \lambda_i = 1 \right\}
\end{equation}

In the context of optimization, this property is useful for formulating constraints, as a constraint on a B-spline function can be enforced by imposing them on the control points \citep{mercy2017spline}. 
\begin{equation}
    a \leq c_i \leq b \quad \forall i\in\{1,2,\ldots,n\} \implies a \leq f(x) \leq b \quad \forall x
\end{equation}
Here $a$ and $b$ are constants of appropriate dimensions and the $\leq$ operator is applied element-wise.

\section{Operations on B-splines}
From the recursive relation in \cref{eq:b-spline-recurrence} it follows that the B-spline basis functions are piecewise polynomial functions in $x$ of degree $p$. This property makes it clear that all polynomial operations on B-splines will result in a new B-spline. Some 


\subsection{Differentiation}
Given a B-spline $f(x) = \sum_{j=0}^{n-1} c_j B_{j, p, \mathbf{t}}(x)$,
the derivative of the B-spline can be computed by simply differencing its coefficients. 
\begin{equation}\label{eq:b-spline-derivative}
    \frac{d}{dx} f(x) = (p-1) \sum_{j=1}^{n-1} \frac{c_j-c_{j-1}}{t_{j+p-1}-t_j} B_{j, p-1, \boldsymbol{\tau}}(x)
\end{equation}
where $\boldsymbol{\tau} = [t_j]_{j=1}^{n+p-1}$ is the same knot vector as $\mathbf{t}$, but with the first and last knot removed in order to maintain \todo{finn ord som passer}.

The new coefficients can be expressed using a transformation matrix $\mathbf T_D$ such that $\mathbf{c}_D = \mathbf T_D \mathbf{c}$, where $\mathbf{c}_D$ is the vector of coefficients of the derivative B-spline. The transformation matrix is defined as an $(n-1) \times n$ matrix with elements:

\begin{equation}
    \mathbf T_D = (p-1) \begin{bmatrix}
        -q_{1,p-1,\mathbf{t}}^{-1} & q_{1,p-1,\mathbf{t}}^{-1} & 0 & \cdots & 0 \\
        0 & -q_{2,p-1,\mathbf{t}}^{-1} & q_{2,p-1,\mathbf{t}}^{-1} & \cdots & 0 \\
        \vdots & \vdots & \vdots & \ddots & \vdots \\
        0 & \cdots & 0 & -q_{n-1,p-1,\mathbf{t}}^{-1} & q_{n-1,p-1,\mathbf{t}}^{-1} 
    \end{bmatrix},
\end{equation}
where $q_{j,p,\mathbf{t}} = (t_{j+p}-t_j)$.


The denominator $t_{j+p}-t_j$ in \cref{eq:b-spline-derivative} illustrates the continuity of the B-spline with respect to knot multiplicity. If a knot is repeated more than $p$ times, the denominator will be zero, and the derivative is not defined at that knot.

\subsection{Integration}
Given a degree $p$ and initial knot vector $\mathbf t$. \cref{eq:b-spline-derivative}, gives
\begin{equation}\label{eq:b-spline-integral-pre}
    p \sum_{j=1}^{n-1} \gamma_j B_{j, p, \boldsymbol{t}}(x) 
    = \frac{d}{dx} \sum_{j=0}^{n-1} c_j B_{j, p+1, \boldsymbol{\tau}}(x),
\end{equation}
given that
\begin{equation}
    \gamma_j = \frac{c_j-c_{j-1}}{t_{j+p}-t_j},
\end{equation}
holds. Here, $\boldsymbol{\tau}$ is now the knot vector $\mathbf{t}$ with 1 additional knot at the beginning and end. Now
the new coefficients on the right hand side of \cref{eq:b-spline-integral-pre} can be expressed as 
\begin{equation}\label{eq:b-spline-integral-coefficients}
    c_j = c_{j-1} + \gamma_j \frac{t_{j+p}-t_j}{p}.
\end{equation}
This then gives the general anti-derivative of a B-spline as
\begin{equation}\label{eq:b-spline-integral}
    \int f(x) dx = \sum_{j=0}^{n-1} c_j B_{j, p+1, \boldsymbol{\tau}}(x)
\end{equation}
where $[c_j]_1^{n-1}$ are given by \cref{eq:b-spline-integral-coefficients} and $c_0$ is the constant of integration and $[\gamma_j]_1^{n-1}$ are the original coefficients of $f(x)$. In matrix form, the coefficients of the integral B-spline can be expressed as
\begin{equation}\label{eq:b-spline-integral-matrix}
    \mathbf{c}_I = \mathbf T_I \mathbf{c} + \mathbf{c}_0,
\end{equation}
where
\begin{equation}
    \mathbf T_I = \frac{1}{p}\begin{bmatrix}
        0 & 0 & \cdots & 0 \\
        q_{1,p,\mathbf{t}} & 0 & \cdots & 0 \\
        q_{1,p,\mathbf{t}} & q_{2,p,\mathbf{t}} & \cdots & 0 \\
        \vdots & \vdots & \ddots & \vdots \\
        q_{1,p,\mathbf{t}} & q_{2,p,\mathbf{t}} & \cdots & q_{n-1,p,\mathbf{t}}
    \end{bmatrix},\quad
    \mathbf{c}_0 = \begin{bmatrix}
        c_0 \\
        c_0 \\
        \vdots \\
        c_0
    \end{bmatrix}
\end{equation}

and, $q_{j,p,\mathbf{t}} = (t_{j+p}-t_j)$ still.

\subsection{Knot Insertion}
In order to perform binary operations on spline functions (e.g. addition), it is convenient to have them be represeted in the same basis, e.g., have the same degree and knot vector. If this is not the case, the B-splines must be transformed to a common form. This motivates the need for operations that can transform B-splines between different bases without changing the shape of the curve.

Knot insertion is, as the name suggests, the operation of extending the knot vector by adding new knots. Given a spline function $f(x) = \mathbf{c}^{\top} \mathbf{B}_{p, \mathbf{t}}(x)$ with knot vector $\mathbf t$, the goal is to express the same function in a new basis $\mathbf{B}_{p, \boldsymbol{\tau}}(x)$ with knot vector $\boldsymbol{\tau}$.

A simple example of a knot vector $\mathbf t$ and a refinement $\boldsymbol \tau$ is given by
$$
    \mathbf t = [0, 0, 0, 1, 2, 3, 5, 5, 5], \quad \boldsymbol \tau = [0, 0, 0, 1, 1, 1, 2, 2, 3, 4, 5, 5, 5].
$$
Here two knots have been added at 1, one at 2, and one at 4.

One can use  a recurrence relation for discrete B-splines to find the transformation matrix.

There are more sophisticated algorithms that avoid computing zero-valued basis functions, such as the Oslo algorithms, but that is beyond the scope of this thesis.

\subsection{Degree Elevation}

\subsection{Addition and Subtraction - Linear}

\subsection{Multiplication of Univariate B-splines}

\subsection{Inner Product}



