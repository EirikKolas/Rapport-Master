%===================================== CHAP 5 =================================

\chapter{Conclusions and Further Work}
\label{chap:conclusions}

This work introduces a unified framework for collision avoidance and trajectory planning of autonomous surface vessels that facilitates compliance with the international regulations for preventing collisions at sea (COLREGS) by leveraging B‐splines for mixed integer nonlinear model‐predictive control (B-MINMPC).  Key features include:

\begin{itemize}
    \item \textbf{COLREGS-facilitating optimal control formulation:}  Casts path following as a finite-dimensional nonlinear program using B-splines for both ship states and time, so that kinematic and dynamic limits become convex or polynomial constraints on spline coefficients. Introduces a novel inverse-Gramian weighting to suppress high-frequency oscillations for the projected–cross-track error (PXTE); a novel reference following metric explained in \cref{sec:reference-following-metric}. The individual objectives of the cost function are weighted by B-splines, allowing for flexible time-varying trade-offs between speed, path-following accuracy and maneuverability without sacrificing convexity in the objective function, enabling COLREGS-compliant behavior.
    
    \item \textbf{Theoretical insights into B-spline parameterization:}  Derives the structural consequences of the B-spline representation, showing that it inherently induces transient deceleration near inflection points. This insight---detailed in \cref{sec:case-2-2-refined-degree-2-basis}---aids in understanding the trade-offs between trajectory smoothness and maneuverability, and provides a theoretical basis for the choice of spline parameterization in maritime applications.

    \item \textbf{Mixed-integer collision modeling:}  Encodes port/starboard passage decisions around multiple moving obstacles via Big–M mixed-integer constraints on separating hyperplanes. A single binary variable per target ship selects the compliant side, efficiently resolving the non-convex collision-avoidance problem in one mixed-integer nonlinear program (MINLP).

    \item \textbf{Open-source B-spline optimization library:}  Delivers a Python/CasADi package implementing all B-spline operations---differentiation, integration, basis transforms, products and inner products---via a general collocation-based coefficient transform, plus high-level APIs for declaring spline variables, mixed-integer constraints and structured costs.

    \item \textbf{Extensive performance analysis:}  Validates the framework across stationary, head-on, crossing, overtaking and confined-waters scenarios. Demonstrates correct COLREGS decisions, smooth trajectory generation, tunable speed-vs-course trade-offs via spline weights, and robust convergence in non-convex environments. Real-time performance is achievable with moderate problem sizes, and with simple modifications to the constraint structure, the problem can be converted from a MINLP to a mixed-integer second-order cone program (MI-SOCP) for even faster solution times and guaranteed optimality convergence.

    \item \textbf{Identification of limitations:}  Highlights challenges such as scenario-specific weight tuning, speed reductions inherent to polynomial spline parameterizations and convergence difficulty under highly nonlinear motion models. 
\end{itemize}


Overall, the B‐spline MINMPC framework achieves smooth, provably safe and COLREGS‐facilitated trajectories with moderate problem sizes and robust convergence in non-convex maritime environments. Precise control over trajectory characteristics via B-spline weights enables flexible trade-offs between speed, path-following accuracy and maneuverability, while the inverse Gramian-based cost function mitigates oscillations without sacrificing maneuverability. 
The thesis also provides new insight into structural effects of B-spline tracking formulations, including a derived explanation for temporary deceleration effects under curvature continuity 
constraints.

The primary strength of the B-MINMPC framework lies in its capacity to generate trajectories that adhere to COLREGS through the adjustment of a limited set of parameters. This feature simplifies the development of COLREGS-compliant motion planners by establishing a clear relationship between parameter settings and the resultant trajectory characteristics. 

Future research will focus on bridging the gap between human-interpretable COLREGS rules and tractable compliance metrics, and on using these metrics to automate the tuning of B-MINMPC parameters. By codifying compliance into measurable criteria, the framework would automatically adjust its weight functions and constraints, which is critical for a real-world implementation of this framework.
Further extensions could include explicit rule-embedding in the optimization formulation, enabling direct COLREGS-conformant motion planning. Real-time performance could be achieved via basis reduction, modifying hyperplane constraints, or converting the MINLP to a convex MI-SOCP for faster solution times and guaranteed optimality convergence.