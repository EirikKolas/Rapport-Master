% !TeX root = main.tex
%===================================== CHAP 2 =================================

\chapter{Background Theory}\label{chap:background-theory}

\section{The B-spline}
\label{sec:b-spline-theory}
This chapter introduces B-splines, piecewise polynomial functions prevalent in computer graphics, CAD, and numerical optimization. It covers their definition, properties, and common operations, emphasizing results and intuition over exhaustive mathematical detail. The aim is to equip the reader with the foundational knowledge required for subsequent chapters and for implementing B-splines in numerical optimization.


\subsection{Why B-splines?}

Trajectory planning requires representing a smooth curve through space, constrained by start and end points, and potentially some desired shape. B-splines are an ideal choice for this, offering a compact and flexible way to describe such curves with a small number of parameters.

B-splines (basis splines) are piecewise polynomial functions defined by:
\begin{itemize}
    \item A \textbf{degree} \( p \)
    \item A \textbf{knot vector} defining where each piece starts and ends
    \item A set of \textbf{control points} (coefficients) that determine the shape
\end{itemize}

The resulting curve is smooth, local, and lies within the convex hull of the control points. This makes it ideal for optimization: constraints on the curve can be translated into constraints on the coefficients.


\cref{fig:b-spline-geometric} illustrate the geometric interpretation of B-splines. In \cref{fig:b-spline-knots-control-points}, a cubic B-spline is shown along with its control polygon, which defines the overall shape of the curve. \Cref{fig:b-spline-convex-hull} emphasizes a key property: the B-spline is always contained within the convex hull of its control points. This property is particularly useful in optimization, as constraints on the curve can be imposed directly on the control points.

\begin{figure}
    \centering
    \begin{subfigure}[b]{0.45\textwidth}
        \centering
        \includesvg[width=\textwidth]{fig/b-spline/knots + control points.svg}
        \caption{Illustration of a cubic B-spline with control points and knot vector. The dashed line represents the control polygon.}
        \label{fig:b-spline-knots-control-points}
    \end{subfigure}
    \hfill
    \begin{subfigure}[b]{0.45\textwidth}
        \centering
        \includesvg[width=\textwidth]{fig/b-spline/convex hull.svg}
        \caption{Convex hull of B-spline. The segment of the B-spline between the third and fourth control points (orange) lies within the convex hull of those points.}
        \label{fig:b-spline-convex-hull} 
    \end{subfigure}
    \caption{Geometric interpretation of B-splines}
    \label{fig:b-spline-geometric}
\end{figure}

Mathematically, a B-spline curve is written as:
\[
f(x) = \sum_{j=0}^{n-1} c_j B_{j,p}(x)
\]
where \( c_j \) are the control points and \( B_{j,p}(x) \) are the B-spline basis functions.

In this thesis, B-splines are used to represent the trajectory of the vessel in a way that is:
\begin{itemize}
    \item \textbf{Differentiable} (for velocity and acceleration)
    \item \textbf{Efficient} (sparse representation)
    \item \textbf{Well-behaved} (satisfies constraints over time)
\end{itemize}

This allows trajectory optimization to be done in continuous time while ensuring global constraint satisfaction. 





\subsection{Definition and Properties}\label{sec:b-spline-definition}
While the previous section gave an intuitive overview of B-splines and their role in trajectory optimization, this section provides the formal mathematical definitions and properties that underlie their behavior.


A B-spline function $f: \mathbb{R} \rightarrow \mathbb{R}$ is defined by $n$ \emph{B-spline coefficients} $\mathbf{c} = [c_j]_{j=0}^{n-1}$ and a non-decreasing sequence of $n+p+1$ \emph{knots} $\mathbf{t} = [t_j]_{j=0}^{n+p}$:

\begin{equation}\label{eq:b-spline-def}
    f(x) = f(x ; \mathbf{c}, p, \mathbf{t})=\sum_{j=0}^{n-1} c_j B_{j, p, \mathbf{t}}(x)=\mathbf{c}^{\top} \mathbf{B}_{\mathrm{p}, \mathrm{t}}(\mathrm{x})
\end{equation}
Where appropriate, the shorter notation $f(x)$ is used when the parameters $\mathbf{c}, p$, and $\mathbf{t}$ are clear from the context. The coefficients $c_j$ determine the shape of the B-spline, and $p$ denotes its degree, a non-negative integer. The $B_{j, p, \mathbf{t}}(x)$ are the B-spline basis functions, defined recursively based on the degree $p$ and the knot vector $\mathbf{t}$. These basis functions form the vector $\mathbf{B}_{p, \mathbf{t}}(x) = [B_{j, p, \mathbf{t}}(x)]_{j=0}^{n-1}$ used in \cref{eq:b-spline-def}. The B-spline basis functions can be defined as \citep{deBoor1978practicalguide}:

\begin{subequations}\label{eq:b-spline-recurrence}
    \begin{align}
        B_{j, p, \mathbf{t}}(x) & =\frac{x-t_j}{t_{j+p}-t_j} B_{j, p-1, \mathrm{t}}(x)+\frac{t_{j+1+p}-x}{t_{j+1+p}-t_{j+1}} B_{j+1, p-1, \mathrm{t}}(x) \label{eq:b-spline-recurrence-p} \\
        B_{j, 0, \mathbf{t}}(x) & := 
        \mathbf 1_{[t_j, t_{j+1})} =
        \begin{cases}
            1, & t_j \leq x<t_{j+1} \\
            0, & \text { otherwise. }
        \end{cases} \label{eq:b-spline-recurrence-0}
    \end{align}
\end{subequations}

From \cref{eq:b-spline-recurrence}, it is clear that the B-spline basis functions are all non-negative, as \cref{eq:b-spline-recurrence-0} gives $B_{j, 0, \mathbf{t}}(x) \geq 0 \quad\forall j, x$, and the coefficients $\frac{x-t_j}{t_{j+p}-t_j}$ and $\frac{t_{j+1+p}-x}{t_{j+1+p}-t_{j+1}}$
in \cref{eq:b-spline-recurrence-p} are also non-negative $\forall x, j\in[0,\dots,n-1], p, \mathbf t$ by the non-decreasing condition on the knots $\mathbf t$. 

\Cref{eq:b-spline-recurrence} also implies $B_{j, p, \mathbf{t}}(x)$ has local support on the interval $[t_j, t_{j+p+1})$. This can be seen by noting that the support of $B_{j, 0, \mathbf{t}}$, $\text{supp}(B_{j, 0, \mathbf{t}}) = [t_j, t_{j+1})$ and for each of the $p$ recurrence steps in \cref{eq:b-spline-recurrence-p}, the support is extended by one knot. 

It can also be shown that the B-spline basis functions are a partition of unity, i.e. $\sum_{j=0}^{n-1} B_{j, p, \mathbf{t}}(x) = 1 \quad\forall x \in [t_0, t_{n+p})$ \citep{deBoor1978practicalguide}.

These properties are well-documented in the literature and are summarized as follows:
\begin{property}[Nonnegativity]\label{b-prop:nonnegativity}
    $B_{j, p, \mathbf{t}}(x) \geq 0 \quad\forall j, p$, and $x$.
\end{property}

\begin{property}[Local support]\label{b-prop:localsupport}
    $B_{j, p, \mathbf{t}}(x)=0 \quad\forall x \notin [t_j, t_{j+p+1})$.
\end{property}

\begin{property}[Partition of unity]\label{b-prop:partitionofunity}
    $\sum_{j=i-p}^i B_{j, p, \mathbf{t}}(x)=1 \quad\forall x \in [t_i, t_{i+1})$.
\end{property}
So far, only B-splines with a one-dimensional co-domain have been considered. However, B-splines can be extended to curves in higher dimensions by creating a vector where each element is a B-spline sharing the same basis. In this case the B-spline is a vector-valued function $\mathbf{f}: \mathbb R \rightarrow \mathbb R^m$, where $\mathbf{f}(x) = [f_1(x), f_2(x), \dots, f_m(x)]^\top$ and each $f_i(x)$ is a B-spline. The \cref{b-prop:nonnegativity,b-prop:localsupport,b-prop:partitionofunity} apply to each $f_i(x)$ individually.



Together, \cref{b-prop:nonnegativity,b-prop:partitionofunity} imply that the B-spline in \cref{eq:b-spline-def} is a convex combination of the coefficients $\mathbf c$. This means that the B-spline is always within the convex hull of the control points $\mathbf c$.

\begin{property}[Convex hull]\label{b-prop:convexhull}
    $f(x) \in \text{conv}(\mathbf{c})
    \quad\forall x$.
\end{property}

The convex hull $\text{conv}(\mathbf{c})$ of a set of points $\mathbf{c} = \{c_1, c_2, \ldots, c_n\}$ is defined as the set of all convex combinations of the points. Mathematically, it is given by:
\begin{equation}
    \text{conv}(\mathbf{c}) = \left\{ \sum_{i=1}^n \lambda_i c_i \mid \lambda_i \geq 0, \sum_{i=1}^n \lambda_i = 1 \right\}
\end{equation}

This also applies to the vector-valued B-spline, which is a powerful property in optimization, as it allows for constraints on the B-spline to be imposed directly on the control points.

The B-spline basis functions, for a given degree \(p\) and knot vector \(\mathbf{t}\), span a linear vector space denoted as \(\mathcal{S}_{p, \mathbf{t}}\). This space, referred to as the spline space, encompasses all piecewise polynomials of degree less than or equal to \(p\) defined over the knot vector \(\mathbf{t}\) \citep{Grimstad2016}. The property of splines spanning a linear vector space is crucial for using B-splines in optimization. 
It allows the spline representation of functions to be converted between various bases by simply applying a matrix transform on its coefficients. This enables efficient computation and manipulation of the B-spline objects.

\subsection{Operations on B-splines}



\subsection{Pythogerean-Hodograph B-splines (PH B-Splines)}\label{sec:pythogerean-hodograph}
The Pythogerean-Hodograph (PH) B-spline is utilized to leverage exact geometric quantities in CAD/CAM. Pythagorean‐Hodograph (PH) B‐splines \cite{Albrecht2016}, a natural extension of PH Bézier curves \cite{Farouki1990}, are employed. In this framework, the hodograph (derivative) 
\begin{equation*}
    \mathbf r'(x) = [a'(x), b'(x)]
\end{equation*}
of a planar B‐spline
\begin{equation*}
    \mathbf r(x) = [a(x), b(x)],
\end{equation*}
satisfies the Pythagorean condition
\begin{equation}\label{eq:pythagorean-condition}
a'(x)^2 + b'(x)^2 = \sigma(x)^2,
\end{equation}
where $\sigma(x)$ is itself a B‐spline function. Enforcing \cref{eq:pythagorean-condition} guarantees that magnitude, curvature, turn‐rate, and arc‐length admit closed‐form rational expressions without numerical approximation. The following derivation is based on \cite{Albrecht2016}.

Introducing the complex pre‐image
\begin{equation*}
    z(x) = u(x) + i\,v(x),
\end{equation*}
the hodograph is defined via the square relation
\begin{equation}\label{eq:hodograph-square}
    \mathbf r'(x) = z(x)^2.
\end{equation}
Expanding the complex square yields
\begin{equation*}
    (u + i v)^2 = u^2 + 2iuv - v^2 = (u^2 - v^2) + i(2uv),
\end{equation*}
and equating real and imaginary parts gives
\begin{subequations}\label{eq:hodograph-derivatives}
    \begin{align}
        a'(x) &= \Re\bigl(z(x)^2\bigr) = u(x)^2 - v(x)^2, \label{eq:hodograph-derivatives-a}\\
        b'(x) &= \Im\bigl(z(x)^2\bigr) = 2\,u(x)\,v(x), \label{eq:hodograph-derivatives-b}
    \end{align}
\end{subequations}
while the magnitude of the hodograph follows as
\begin{equation*}
    \sigma(x) = \|\mathbf r'(x)\| = |z(x)|^2 = u(x)^2 + v(x)^2.
\end{equation*}
Finally, integrating recovers the curve coordinates:
\begin{equation*}
    \begin{aligned}
        a(x) &= a_0 + \int_{x_0}^{x}\bigl(u(\tau)^2 - v(\tau)^2\bigr)\,d\tau,\\
        b(x) &= b_0 + \int_{x_0}^{x}2\,u(\tau)\,v(\tau)\,d\tau,
    \end{aligned}
\end{equation*}

where $(a_0,b_0)=\mathbf r(x_0)$.

The curvature $\kappa(x)$ of the curve $\mathbf r(x)$ is defined as
\begin{equation*}
    \kappa(x) = \frac{a'(x)b''(x) - b'(x)a''(x)}{\|\mathbf r'(x)\|^3},
\end{equation*}
where $a'(x)$ and $b'(x)$ are the first derivatives of the curve, and $a''(x)$ and $b''(x)$ are the second derivatives. Using \cref{eq:hodograph-derivatives} gives
which in terms of $u(x)$ and $v(x)$ simplifies to
\begin{equation}\label{eq:curvature}
    \kappa(x) = \frac{2(uv'-vu')}{(u^2+v^2)^2}.
\end{equation}
The turn-rate $\dot\theta(x)$ is then
\begin{equation}\label{eq:turn-rate}
    \dot\theta(x) = \kappa(x) \sigma(x) = \frac{2(uv'-vu')}{u^2+v^2},
\end{equation}
while the arc-length $s(x)$ is given by
\begin{equation*}
    s(x) = \int_{x_0}^{x}\sigma(\tau)\,d\tau.
\end{equation*}


\subsection{Non-Uniform Rational B-splines (NURBS)}
In the previous section, rational functions of B-splines were mentioned in the context of Pythagorean-Hodograph curves. \Cref{eq:curvature,eq:turn-rate} are examples of such rational functions. This section formalizes the concept of Non-uniform rational B-splines (NURBS), for which a good reference is \cite{Piegl1997}. 

NURBS are defined by the rational basis functions
\begin{equation}\label{eq:nurbs-basis}
R_{i,p,\mathbf t, \mathbf w}(x)  = \frac{w_i,B_{i,p,\mathbf t}(x)}{\sum_{j=0}^{n-1}w_j,B_{j,p,\mathbf t}(x)},
\end{equation}
for which a NURBS curve is given by
\begin{equation}\label{eq:nurbs-curve}
\mathbf r(x; \mathbf c, p, \mathbf t, \mathbf w)  = \sum_{i=0}^{n-1}R_{i,p,\mathbf t, \mathbf w}(x)\,c_i
= \mathbf R_{p,\mathbf t, \mathbf w}(x)^\top \mathbf c.
\end{equation}

NURBS inherit the local support and non-negativity properties from B-splines. When all weights $w_i$ are equal, the NURBS reduces exactly to a B-spline curve. The partition of unity property is also preserved as long as all the weights are non-negative. \citep{Piegl1997}

A principal advantage of NURBS is their ability to represent conic sections---such as circles, ellipses, and parabolas---exactly using quadratic or cubic segments \citep{Piegl1997}. For example, a circle can be constructed by appropriate choice of four rational quadratic segments, each representing a quarter arc \citep{DenbighStarkeyNURBS}. Many CAD systems likewise rely on NURBS for the exactness of the conic-surface representation as well as the ability to create flexible free-form shapes \citep{Farin1991,PieglTillerSIGGRAPH,cottrell2009isogeometric}.

In \cref{sec:b-spline-definition} it is mentioned that a B-spline function $f(x; p, \mathbf t)$ is part of the linear vector space $\mathbb S_{p, \mathbf t}$.
, a NURBS function $f(x; p, \mathbf t, \mathbf w)$ is part of the rational spline space $\mathbb B_{p, \mathbf t, \mathbf w}$. This space contains all piecewise rational polynomials on the knot vector $\mathbf t$ with degree $p$ and weights $\mathbf w$. 



\subsection{Summary}

This chapter has introduced the concept of B-splines, their properties, and how they can be used to represent curves and surfaces. B-splines are piecewise polynomial functions defined over a knot vector, and they have several important properties, including local support, non-negativity, and the ability to represent complex shapes. The chapter also discussed the operations of addition, multiplication, and inner product for B-splines, as well as their generalization to Non-Uniform Rational B-splines (NURBS).

Table of operations for B-splines is given in \cref{tab:operations}. A summary of operations on \acrshort{NURBS} can be found in \cite{piegl1997symbolic}.


\renewcommand{\arraystretch}{1.2}
\begin{table}
    \centering
    \small
    \begin{tabular}{|l|l|l|l|c|}
    \hline
    \textbf{Operation} 
      & \textbf{Expression} 
        & \textbf{Space} 
          & \textbf{Coefficients} 
            & \textbf{Implementation} \\
    \hline
    \hline
    Evaluation   
      & $\mathbf f(x)$ 
        & $\mathbb S^n_{p}\to\mathbb{R}^n$ 
          & Linear 
            & \cref{eq:b-spline-recurrence} \\
    \hline
    Derivative   
      & $\mathbf f'$  
        & $\mathbb S^n_{p}\to\mathbb S^n_{p-1}$ 
          & Linear 
            & \cref{eq:b-spline-derivative} \\
    \hline
    Integral     
      & \rule{0pt}{4ex}$\displaystyle\int \mathbf f(x)\,dx$ 
        & $\mathbb S^n_{p}\to\mathbb S^n_{p+1}$ 
          & Linear 
            & \cref{eq:b-spline-integral} \\[1.5ex]
    \hline
    Degree elevation  
      & $\mathbf f$  
        & $\mathbb S^n_{p}\to\mathbb S^n_{p+1}$ 
          & Linear 
            & \Cref{alg:degree-elevation} \\      
    \hline
    Knot insertion    
      & $\mathbf f$  
        & $\mathbb S^n_{p}\to\mathbb S^n_{p}$ 
          & Linear 
            & \Cref{alg:knot-refinement} \\
    \hline
    \hline
    Vector Addition  
      & $\mathbf a + \mathbf f$  
        & $\mathbb R^n \times\mathbb S^n_{p}\to\mathbb S^n_{p}$ 
          & Linear 
            & \Cref{alg:addition} \\
    \hline
    Scalar multiplication  
      & $a\,\mathbf f$  
        & $\mathbb R\times\mathbb S^n_{p}\to\mathbb S^n_{p}$ 
          & Linear 
            & $-$ \\
    \hline
    Addition     
      & $\mathbf f +\mathbf g$  
        & $\mathbb S^n_{p_1}\times\mathbb S^n_{p_2}\to\mathbb S^n_{\max(p_1,p_2)}$ 
          & Linear 
            & \Cref{alg:addition} \\[.5ex]
    \hline
    Multiplication   
      & $\mathbf f\,g$  
        & $\mathbb S^n_{p_1}\times\mathbb S_{p_2}\to\mathbb S^n_{p_1+p_2}$ 
          & Bilinear 
            & \Cref{alg:multiplication} \\
    \hline
    PW inner product  
      & $\mathbf f \cdot \mathbf g$  
        & $\mathbb S^n_{p_1}\times\mathbb S^n_{p_2}\to\mathbb S_{p_1+p_2}$
          & Bilinear 
            & \cref{eq:dot-product-pointwise} \\
    \hline
    Inner product    
      & $\langle \mathbf f, \mathbf g \rangle$  
        & $\mathbb S^n_{p_1}\times\mathbb S^n_{p_2}\to\mathbb R$ 
          & Bilinear 
            & \cref{eq:dot-product} \\
    \hline
    Division  
      & \rule{0pt}{4ex}$\displaystyle\frac{\mathbf f}{g}$  
        & $\mathbb S^n_{p_1}\times\mathbb S_{p_2}\to\mathbb B^n_{\max(p_1,p_2)}$
          & Rational 
            & \Cref{alg:nurbs-conversion} \\[1.5ex]
    \hline
    \end{tabular}
    \caption{Summary of B-spline operations and their properties. The symbols $f$ and $g$ denote B-spline functions, while $a$ denotes a scalar. Their corresponding bold-faced symbols denote vector-valued functions and vectors in $\mathbf R^n$ space respectively. The ``Coefficients'' column indicates the transformation type of the operation on the coefficients. The ``Space'' column indicates the space of the resulting B-spline. The knot vectors for these spaces are ommited for brevity and can be found in the references in the ``Algorithms'' column.}
    \label{tab:operations}
\end{table}
\renewcommand{\arraystretch}{1}
    