% !TeX root = main.tex
%=================================== APPENDIX ===============================


\chapter{Homogeneous Integration Matrix}\label{app:homogeneous-integration-matrix}

To correctly define the integration operator $\mathbf{T}_I$ in matrix form while preserving degrees of freedom, we extend the system using homogeneous coordinates. This ensures that integration does not lose information due to the absence of a known constant term.

\section{Homogeneous Representation of Integration}

In \cref{eq:b-spline-integral-matrix}, the coefficients of the integral B-spline are expressed as
\begin{equation}
    \mathbf{c}_I = \mathbf T_I \mathbf{c}_D + \mathbf{c}_0.
\end{equation}
By prepending the coefficient vector $\mathbf{c}_D$ with 1, we can write this whole expression in homogeneous form as
\begin{equation}
    \mathbf{c}_I = 
    \underbrace{
        \begin{bmatrix}
            \mathbf{c}_0 & \mathbf T_I
        \end{bmatrix} 
    } _{\mathbf{\tilde T}_I}
    \underbrace{
        \begin{bmatrix}
            1 \\
            \mathbf{c}_D
        \end{bmatrix}
    }_{\mathbf{\tilde c}_D}.
\end{equation}

By inverting the transformation matrix $\mathbf{\tilde T}_I$, we can express the derivative coefficients as
\begin{equation}
    \mathbf{\tilde c}_D = \mathbf{\tilde T}_I^{-1} \mathbf{c}_I.
\end{equation}

Proof:
