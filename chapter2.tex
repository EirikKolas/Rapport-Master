% !TeX root = main.tex
%===================================== CHAP 2 =================================

\chapter{Background Theory}
\label{chap:background-theory}

\section{B-splines}

\subsection{Definition}
The following definition is from \cite{Grimstad2016}:
A B-spline $f: \mathbb R \rightarrow \mathbb R$ is given by $n$ \emph{B-spline coefficients} $\mathbf c = [c_j]_{j=0}^{n-1}$ and $n+p+1$ knots $\mathbf t = [t_j]_{j=0}^{n+p}$ as

\begin{equation}\label{eq:b-spline-def}
    f(x ; \mathbf{c}, p, \mathbf{t})=\sum_{j=0}^{n-1} c_j B_{j, p, \mathbf{t}}(x)=\mathbf{c}^{\top} \mathbf{B}_{\mathrm{p}, \mathrm{t}}(\mathrm{x})
\end{equation}

When the parameters $\mathbf{c}, p$, and $\mathbf{t}$ are given by the context $f(x ; \mathbf{c}, p, \mathbf{t})$ is simply denoted $f(x)$. In \cref{eq:b-spline-def}, $\mathbf{B}_{p, \mathbf{t}}(x)=\left[B_{j, p, \mathbf{t}}(x)\right]_{j=0}^{n-1}$ is a column vector of $p$ th-degree B-spline basis functions, defined by the recurrence relation

\begin{equation}    
    \begin{aligned}
        B_{j, p, \mathbf{t}}(x) & =\frac{x-t_j}{t_{j+p}-t_j} B_{j, p-1, \mathrm{t}}(x)+\frac{t_{j+1+p}-x}{t_{j+1+p}-t_{j+1}} B_{j+1, p-1, \mathrm{t}}(x) \\
        B_{j, 0, \mathbf{t}}(x) & = 
        \begin{cases}
            1, & t_j \leq x<t_{j+1} \\
            0, & \text { otherwise }
        \end{cases}
    \end{aligned}
\end{equation}


\subsection{Useful Properties}
Property 1

(Nonnegativity) $B_{j, p, \mathbf{t}}(x) \geq 0$ for all $j, p$, and $x$.

Property 2

(Local support) $B_{j, p, \mathbf{t}}(x)=0$ for all $x \notin\left[t_j, t_{j+p+1}\right.)$.

Property 3

(Partition of unity) $\sum_{j=i-p}^i B_{j, p, \mathbf{t}}(x)=1$ for all $x \in\left[t_i, t_{i+1}\right.)$.


\subsection{Operations on B-splines}

Derivative

Anti-Derivative and Integrals

Knot Insertion and Degree Elevation

Addition ans subtraction - Linear

Multiplication of univariate B-splines

Inner Product


These operations allows for the formulation of constraints $\mathbf g(x)$ that exploit the convex hull property to gaurantee constraint-satisfaction for all $x$.

\section{Semi-infinite Programming}
