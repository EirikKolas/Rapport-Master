% !TeX root = main.tex
%===================================== CHAP 2 =================================

\chapter{Background Theory}\label{chap:background-theory}
In order to establish the theoretical foundations for the B-spline MINMPC framework, key topics in spline theory, maritime collision regulations, and optimal control are presented. First, the definition, properties, and fundamental operations of B-splines are reviewed to motivate their use in continuous-time trajectory representation and constraint formulation. Then, the relevant COLREGS rules and vessel-to-vessel encounter types are summarized to guide the design of collision-avoidance constraints. Finally, the principles of semi-infinite programming, multiple shooting, and spline-based relaxation in optimal control are introduced to frame the trajectory optimization method. These elements together provide the essential background for the development and analysis in subsequent chapters.



\section{The B-spline}
\label{sec:b-spline-theory}
This chapter introduces B-splines, piecewise polynomial functions prevalent in computer graphics, CAD, and numerical optimization. It covers their definition, properties, and common operations, emphasizing results and intuition over exhaustive mathematical detail. The aim is to equip the reader with the foundational knowledge required for subsequent chapters and for implementing B-splines in numerical optimization.


\subsection{Why B-splines?}

Trajectory planning requires representing a smooth curve through space, constrained by start and end points, and potentially some desired shape. B-splines are an ideal choice for this, offering a compact and flexible way to describe such curves with a small number of parameters.

B-splines (basis splines) are piecewise polynomial functions defined by:
\begin{itemize}
    \item A \textbf{degree} \( p \)
    \item A \textbf{knot vector} defining where each piece starts and ends
    \item A set of \textbf{control points} (coefficients) that determine the shape
\end{itemize}

The resulting curve is smooth, local, and lies within the convex hull of the control points. This makes it ideal for optimization: constraints on the curve can be translated into constraints on the coefficients.


\cref{fig:b-spline-geometric} illustrate the geometric interpretation of B-splines. In \cref{fig:b-spline-knots-control-points}, a cubic B-spline is shown along with its control polygon, which defines the overall shape of the curve. \Cref{fig:b-spline-convex-hull} emphasizes a key property: the B-spline is always contained within the convex hull of its control points. This property is particularly useful in optimization, as constraints on the curve can be imposed directly on the control points.

\begin{figure}
    \centering
    \begin{subfigure}[b]{0.45\textwidth}
        \centering
        \includesvg[width=\textwidth]{fig/b-spline/knots + control points.svg}
        \caption{Illustration of a cubic B-spline with control points and knot vector. The dashed line represents the control polygon.}
        \label{fig:b-spline-knots-control-points}
    \end{subfigure}
    \hfill
    \begin{subfigure}[b]{0.45\textwidth}
        \centering
        \includesvg[width=\textwidth]{fig/b-spline/convex hull.svg}
        \caption{Convex hull of B-spline. The third B-spline segment (blue) lies within the convex hull of the points $\{\mathbf c_1,\ldots,\mathbf c_4\}$ (orange).}
        \label{fig:b-spline-convex-hull} 
    \end{subfigure}
    \caption{Geometric interpretation of B-splines}
    \label{fig:b-spline-geometric}
\end{figure}

Mathematically, a B-spline curve is written as:
\[
f(x) = \sum_{j=0}^{n-1} c_j B_{j,p}(x)
\]
where \( c_j \) are the control points and \( B_{j,p}(x) \) are the B-spline basis functions.

In this thesis, B-splines are used to represent the trajectory of the vessel in a way that is:
\begin{itemize}
    \item \textbf{Differentiable} (Conftinuuous derivatives up to order \( p \))
    \item \textbf{Efficient} (sparse representation)
    \item \textbf{Well-behaved} (Local support and convex hull properties)
\end{itemize}

This allows a continuous-time trajectory optimization problems to be solved while ensuring global constraint satisfaction. 





\subsection{Definition and Properties}\label{sec:b-spline-definition}
While the previous section gave an intuitive overview of B-splines and their role in trajectory optimization, this section provides the formal mathematical definitions and properties that underlie their behavior.


A B-spline function $f: \mathbb{R} \rightarrow \mathbb{R}$ is defined by $n$ \emph{B-spline coefficients} $\mathbf{c} = [c_j]_{j=0}^{n-1}$ and a non-decreasing sequence of $n+p+1$ \emph{knots} $\mathbf{t} = [t_j]_{j=0}^{n+p}$:

\begin{equation}\label{eq:b-spline-def}
    f(x) = f(x ; \mathbf{c}, p, \mathbf{t})=\sum_{j=0}^{n-1} c_j B_{j, p, \mathbf{t}}(x)=\mathbf{c}^{\top} \mathbf{B}_{\mathrm{p}, \mathrm{t}}(\mathrm{x})
\end{equation}
Where appropriate, the shorter notation $f(x)$ is used when the parameters $\mathbf{c}, p$, and $\mathbf{t}$ are clear from the context. The coefficients $c_j$ determine the shape of the B-spline, and $p$ denotes its degree, a non-negative integer. The $B_{j, p, \mathbf{t}}(x)$ are the B-spline basis functions, defined recursively based on the degree $p$ and the knot vector $\mathbf{t}$. These basis functions form the vector $\mathbf{B}_{p, \mathbf{t}}(x) = [B_{j, p, \mathbf{t}}(x)]_{j=0}^{n-1}$ used in \cref{eq:b-spline-def}. The B-spline basis functions can be defined using the Cox-de Boor recurrence \citep{deBoor1978practicalguide}:

\begin{subequations}\label{eq:b-spline-recurrence}
    \begin{align}
        B_{j, p, \mathbf{t}}(x) & =\frac{x-t_j}{t_{j+p}-t_j} B_{j, p-1, \mathrm{t}}(x)+\frac{t_{j+1+p}-x}{t_{j+1+p}-t_{j+1}} B_{j+1, p-1, \mathrm{t}}(x) \label{eq:b-spline-recurrence-p} \\
        B_{j, 0, \mathbf{t}}(x) & := 
        \mathbf 1_{[t_j, t_{j+1})} =
        \begin{cases}
            1, & t_j \leq x<t_{j+1} \\
            0, & \text { otherwise. }
        \end{cases} \label{eq:b-spline-recurrence-0}
    \end{align}
\end{subequations}

% From \cref{eq:b-spline-recurrence}, it is clear that the B-spline basis functions are all non-negative, as \cref{eq:b-spline-recurrence-0} gives $B_{j, 0, \mathbf{t}}(x) \geq 0 \quad\forall j, x$, and the expressions $\frac{x-t_j}{t_{j+p}-t_j}$ and $\frac{t_{j+1+p}-x}{t_{j+1+p}-t_{j+1}}$
% in \cref{eq:b-spline-recurrence-p} are also non-negative $\forall x, j\in[0,\dots,n-1], p, \mathbf t$ by the non-decreasing condition on the knots $\mathbf t$. 

% \Cref{eq:b-spline-recurrence} also implies $B_{j, p, \mathbf{t}}(x)$ has local support on the interval $[t_j, t_{j+p+1})$. This can be seen by noting that the support of $B_{j, 0, \mathbf{t}}$, $\text{supp}(B_{j, 0, \mathbf{t}}) = [t_j, t_{j+1})$ and for each of the $p$ recurrence steps in \cref{eq:b-spline-recurrence-p}, the support is extended by one knot. 

% It can also be shown that the B-spline basis functions are a partition of unity, i.e. 
% $$
% \sum_{j=0}^{n-1} B_{j, p, \mathbf{t}}(x) = 1 \quad\forall x \in [t_0, t_{n+p})$$ 
% \citep{deBoor1978practicalguide}.

% These properties are well-documented in the literature \citep{deBoor1978practicalguide} and are summarized as follows:

The B-spline basis functions $B_{j, p, \mathbf{t}}(x)$ have several important properties which are well documented in the literature \citep{deBoor1978practicalguide}. These properties are summarized as follows:
\begin{property}[Nonnegativity]\label{b-prop:nonnegativity}
    $B_{j, p, \mathbf{t}}(x) \geq 0 \quad\forall j, p$, and $x$.
\end{property}

\begin{property}[Local support]\label{b-prop:localsupport}
    $B_{j, p, \mathbf{t}}(x)=0 \quad\forall x \notin [t_j, t_{j+p+1})$.
\end{property}

\begin{property}[Partition of unity]\label{b-prop:partitionofunity}
    $\sum_{j=i-p}^i B_{j, p, \mathbf{t}}(x)=1 \quad\forall x \in [t_i, t_{i+1})$.
\end{property}

So far, only B-splines with a one-dimensional co-domain have been considered. However, B-splines can be extended to curves in higher dimensions by creating a vector where each element is a B-spline sharing the same basis. In this case the B-spline is a vector-valued function $\mathbf{f}: \mathbb R \rightarrow \mathbb R^d$, where $\mathbf{f}(x) = [f_1(x), f_2(x), \dots, f_d(x)]^\top$ and each $f_i(x)$ is a B-spline. The \cref{b-prop:nonnegativity,b-prop:localsupport,b-prop:partitionofunity} apply to each $f_i(x)$ individually. 


Together, \cref{b-prop:nonnegativity,b-prop:partitionofunity} imply that the B-spline in \cref{eq:b-spline-def} is a convex combination of the coefficients $\mathbf c$. This means that the B-spline is always within the convex hull of the control points $\mathbf c$, which can be formally stated as follows:
\begin{property}[Convex hull]\label{b-prop:convexhull}
    $f(x) \in \text{conv}(\mathbf{c})
    \quad\forall x$.
\end{property}

The convex hull $\text{conv}(\mathbf{c})$ of a set of points $\mathbf{c} = \{c_1, c_2, \ldots, c_n\}$ is defined as the set of all convex combinations of the points. Mathematically, it is given by:
\begin{equation}
    \text{conv}(\mathbf{c}) = \left\{ \sum_{i=1}^n \lambda_i c_i \mid \lambda_i \geq 0, \sum_{i=1}^n \lambda_i = 1 \right\}
\end{equation}
This is a powerful property in optimization, as it allows for constraints on vector-valued B-splines to be imposed directly on the control points.


\subsection{Fundamental B-spline Operations}
\label{sec:b-spline-operations-theory}
The B-spline basis functions, for a given degree $p$ and knot vector $\mathbf{t}$, span a linear vector space  as $\mathcal{S}_{p, \mathbf{t}}$. This space, referred to as the spline space, encompasses all piecewise polynomials of degree less than or equal to $p$ defined over the knot vector $\mathbf{t}$ \citep{Grimstad2016}. The property of splines spanning a linear vector space is crucial for using B-splines in optimization. 
It allows the spline representation of functions to be converted between various bases by simply applying a matrix transform on its coefficients. This enables efficient computation and manipulation of the B-spline objects.

Because the spline space $\mathcal{S}_{p,\mathbf{t}}$ is a vector space, any operation that is linear on the spline itself can be carried out by a suitable matrix acting on those coefficients.  While the following operations---such as differentiation and integration---are important for understanding the application of B-splines in optimization, their practical implementation and role in the cost function and constraint construction are described in detail in \cref{sec:operations-on-b-splines}.

First, the \emph{sum} of two scalar splines with coefficient vectors $\mathbf{c},\mathbf{d}\in\mathbb R^n$ (same degree $p$ and knot vector $\mathbf{t}$) is given by
\begin{equation}
    \begin{aligned}
        f(x) &= \mathbf{B}_{p,\mathbf{t}}^\top(x)\,\mathbf{c}
        \quad\text{and}\quad
        g(x) = \mathbf{B}_{p,\mathbf{t}}^\top(x)\,\mathbf{d} \\
        \quad\Longrightarrow\quad
        (f+g)(x)
        &= \mathbf{B}_{p,\mathbf{t}}^\top(x)\,(\mathbf{c}+\mathbf{d})\!,
    \end{aligned}
\end{equation}
so that addition of the two splines reduces to a sum of their coefficient vectors.  

Second, \emph{differentiation} is also a linear map.  There exists a sparse \emph{derivative matrix} $\mathbf{D}\in\mathbb R^{n\times n}$ such that
\begin{equation}
  \frac{\mathrm{d}}{\mathrm{d}x}\bigl(\mathbf{B}_{p,\mathbf{t}}^\top(x)\,\mathbf{c}\bigr)
  = \mathbf{B}_{p-1,\mathbf{t}}^\top(x)\,\bigl(\mathbf{D}\,\mathbf{c}\bigr).
\end{equation}
Here $\mathbf{D}$ encodes the Cox–de Boor recurrence in \cref{eq:b-spline-recurrence} for the basis‐function derivatives; no manipulation of the analytic form is required beyond this coefficient‐level multiplication. The same is also true for computing the antiderivative, which is a linear map from $\mathcal{S}_{p,\mathbf{t}}$ to $\mathcal{S}_{p+1,\mathbf{t}}$.

These simple examples demonstrate how addition, differentiation, and integration all \emph{condense} to matrix operations on the control‐point coefficients, leveraging the linearity of the spline basis (in particular the partition‐of‐unity property, \cref{b-prop:partitionofunity}) without ever needing to manipulate piecewise‐polynomial formulas directly.


\subsection{Pythogerean-Hodograph B-splines (PH B-Splines)}\label{sec:pythogerean-hodograph}
In motion planning and trajectory optimization, it is often necessary to compute geometric quantities such as curvature, turn-rate, and arc-length. These quantities are essential for ensuring smooth and feasible trajectories. However, computing these quantities from standard B-splines does not yield closed-form expressions in the spline space $\mathcal S$ 

Pythagorean‐Hodograph (PH) B‐splines \citep{Albrecht2016}, which generalize PH Bézier curves \citep{Farouki1990}, form a subclass of B‐splines whose hodograph (derivative) satisfies a Pythagorean relation.  By enforcing 
\[
  \|\mathbf r'(x)\|^2 = a'(x)^2 + b'(x)^2 = \sigma(x)^2,
\]
for a B-spline $\mathbf r(x) = [a(x), b(x)]$ with \(\sigma(x)\) itself a spline, PH B‐splines admit closed‐form expressions for curvature, turn‐rate, and arc‐length directly in the spline space.  

Introducing the scalar spline functions $u(x)$ and $v(x)$, the PH B-spline $\mathbf r(x)$ can be expressed as:
\begin{equation}\label{eq:ph-b-spline}
    \mathbf r(x) = \mathbf r(x_0) + \int_{x_0}^x 
    \begin{bmatrix}
        u(t)^2 - v(t)^2 \\
        2u(t)v(t)
    \end{bmatrix} \mathrm{d}t
\end{equation}
where $x_0$ is the starting point of the B-spline. This form allows for the computation of the curvature and turn-rate as functions of $u(x)$ and $v(x)$ and is used in this paper to parameterize Dubins model (see \cref{sec:dubins-model}). A derivation of this form and the properties of PH B-splines can be found in \cref{app:ph-spline-derivation}.

\subsection{Non-Uniform Rational B-splines (NURBS)}
Another useful construct in the context of B-splines is the Non-Uniform Rational B-spline (NURBS). NURBS generalize B-splines by introducing a set of \emph{weights} associated with each control point, allowing for more flexibility in shape representation.
This section formalizes the concept of Non-uniform rational B-splines (NURBS), for which a good reference is \cite{Piegl1997}. 

NURBS are defined by the rational basis functions
\begin{equation}\label{eq:nurbs-basis}
R_{i,p,\mathbf t, \mathbf w}(x)  = \frac{w_i,B_{i,p,\mathbf t}(x)}{\sum_{j=0}^{n-1}w_j,B_{j,p,\mathbf t}(x)},
\end{equation}
for which a NURBS curve is given by
\begin{equation}\label{eq:nurbs-curve}
\mathbf r(x; \mathbf c, p, \mathbf t, \mathbf w)  = \sum_{i=0}^{n-1}R_{i,p,\mathbf t, \mathbf w}(x)\,c_i
= \mathbf R_{p,\mathbf t, \mathbf w}(x)^\top \mathbf c.
\end{equation}

NURBS inherit the local support, partition of unity and non-negativity properties from B-splines. When all weights $w_i$ are equal, the NURBS reduces exactly to a B-spline curve. \citep{Piegl1997}

A principal advantage of NURBS is their ability to represent conic sections---such as circles, ellipses, and parabolas---exactly using quadratic or cubic segments \citep{Piegl1997}. For example, a circle can be constructed by appropriate choice of four rational quadratic segments, each representing a quarter arc \citep{DenbighStarkeyNURBS}. Many CAD systems likewise rely on NURBS for the exactness of the conic-surface representation as well as the ability to create flexible free-form shapes \citep{Farin1991,PieglTillerSIGGRAPH,cottrell2009isogeometric}.

In \cref{sec:b-spline-definition} it is mentioned that a B-spline function $f(x; p, \mathbf t)$ is part of the linear vector space $\mathcal S_{p, \mathbf t}$. A NURBS function $f(x; p, \mathbf t, \mathbf w)$ is part of the rational spline space $\mathcal B_{p, \mathbf t, \mathbf w}$. This space contains all piecewise rational polynomials on the knot vector $\mathbf t$ with degree $p$ and weights $\mathbf w$. 


