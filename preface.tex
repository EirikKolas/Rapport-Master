% !TeX root = main.tex

\section*{Preface}
\addcontentsline{toc}{chapter}{Preface}
\vspace{0.5cm}

This thesis is the result of my work during the final semester of my Master's degree at the Department of Engineering Cybernetics, Norwegian University of Science and Technology (NTNU) in Trondheim, Norway. The work presented here is the result of a rewarding deepdive into the field of trajectory planning in maritime environments, combining it with my newfound interest in mixed-integer programming and B-spline parametrizations. 
Researching and learning about the challenges in this field has been a overwhelming and humbling experience. Especially the complexity of working with the international regulations for preventing collisions at sea (COLREGS) in the context of autonomous surface vessels has been a fascinating journey.
I am grateful for the opportunity to have worked on this project, and I hope it will be of use to others.

I would like to express my gratitude to my supervisors Morten Breivik and Emil Thyri, who have provided invaluable guidance and support throughout the work on this thesis and the project specialization report. A special thanks goes to Vortex NTNU for the great and inclusive learning environment, and for everything I have learned during my time there. It was this environment that sparked my interest in the maritime domain and gave me the opportunity to work on this project. I would also like to thank my fellow students and friends for their support and encouragement throughout my studies. My time here in Trondheim would not have been the same without them. 

The goal of this thesis is to present a novel framework for trajectory planning in maritime environments that facilitates compliance with COLREGS, while also being flexible and modular enough to be used in a variety of scenarios. 
\begin{itemize}
    \item During the semester, I had follow-up meetings every two weeks with my supervisors, where they contributed guidance and the progress of the work was discussed.
    \item This project started as a continuation of my project specialization report, which was written in the fall of 2024, but the scope has since shifted to the point where it is now a standalone thesis. 
    \item To develop the framework and ease the implementation, a significant amount of time was spent on creating the B-spline optimization library presented in this thesis, which is available as an open-source package on GitHub. 
\end{itemize}    


\begin{center}
    \vspace{1cm}
    Eirik Kolås\\
    Trondheim, June 16, 2025
\end{center}


\cleardoublepage