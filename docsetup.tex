% !TeX root = main.tex
\usepackage[parfill]{parskip}
\usepackage{setspace}
\usepackage[export]{adjustbox}
\usepackage{graphicx}
\usepackage{amssymb}
\usepackage{mathrsfs}
\usepackage{amsthm}
\usepackage{amsmath}
\usepackage{color}
\usepackage{xargs}
\usepackage{subcaption}
\usepackage[pdftex,dvipsnames]{xcolor}
\usepackage{transparent}
\usepackage{svg}
\usepackage[colorinlistoftodos,prependcaption,textsize=small]{todonotes}
\usepackage[Lenny]{fncychap}
\usepackage[pdftex,bookmarks=true]{hyperref}
\usepackage[pdftex]{hyperref}
\hypersetup{
    colorlinks,%
    citecolor=black,%
    filecolor=black,%
    linkcolor=black,%
    urlcolor=black
}
\usepackage[capitalize]{cleveref}
\crefname{equation}{}{}
\crefname{figure}{Figure}{figures}
\crefname{section}{Section}{sections}
\usepackage[font=small,labelfont=bf]{caption}
\usepackage{fancyhdr}
\usepackage{times}
%\usepackage[intoc]{nomencl}
%\renewcommand{\nomname}{List of Abbreviations}
%\makenomenclature
\usepackage{natbib}
\usepackage[style=altsuper4col,acronym]{glossaries}
% \usepackage[style=altsuper4colheader]{glossaries}
\usepackage{float}
\usepackage[section,below,above]{placeins} % Prevent floats from floating into the next section
%\floatstyle{boxed} 
\restylefloat{figure}

% ShareLaTeX does not support glossaries now. Sorry...
%\usepackage[number=none]{glossary}
%\makeglossary
%\newglossarytype[abr]{abbr}{abt}{abl}
%\newglossarytype[alg]{acronyms}{acr}{acn}
%\newcommand{\abbrname}{Abbreviations} 
%\newcommand{\shortabbrname}{Abbreviations}
%%\makeabbr
\newcommand{\HRule}{\rule{\linewidth}{0.5mm}}

\renewcommand*\contentsname{Table of Contents}

\pagestyle{fancy}
\fancyhf{}
\renewcommand{\chaptermark}[1]{\markboth{\chaptername\ \thechapter.\ #1}{}}
\renewcommand{\sectionmark}[1]{\markright{\thesection\ #1}}
\renewcommand{\headrulewidth}{0.1ex}
\renewcommand{\footrulewidth}{0.1ex}
\fancypagestyle{plain}{\fancyhf{}\fancyfoot[LE,RO]{\thepage}\renewcommand{\headrulewidth}{0ex}}

\makeglossaries

\DeclareMathOperator{\asin}{asin}
\DeclareMathOperator{\acos}{acos}
\DeclareMathOperator{\atan}{atan}
\DeclareMathOperator{\sign}{sign}
\DeclareMathOperator*{\argmin}{arg\,min}

\newcommandx{\unsure}[2][1=]{\todo[linecolor=red,backgroundcolor=red!25,
bordercolor=red,#1]{#2}} \newcommandx{\change}[2][1=]{\todo[linecolor=blue,
backgroundcolor=blue!25,bordercolor=blue,#1]{#2}}
\newcommandx{\info}[2][1=]{\todo[linecolor=OliveGreen,backgroundcolor=OliveGreen!
25,bordercolor=OliveGreen,#1]{#2}}
\newcommandx{\improvement}[2][1=]{\todo[linecolor=Plum,backgroundcolor=Plum!25,
bordercolor=Plum,#1]{#2}} 
\newcommandx{\emil}[2][1=]{\todo[linecolor=LimeGreen,backgroundcolor=LimeGreen!25,
bordercolor=LimeGreen,#1]{#2}}
\newcommandx{\morten}[2][1=]{\todo[linecolor=ForestGreen,backgroundcolor=ForestGreen!25,
bordercolor=ForestGreen,#1]{#2}}


%% Algorithm
\usepackage[T1]{fontenc}
\usepackage[utf8]{inputenc} % set input encoding
\usepackage{varwidth}

\usepackage{newfloat}
\DeclareFloatingEnvironment[placement=htp]{algorithm}
\usepackage{algpseudocodex}

\usepackage{caption}
\let\globalcaption=\caption
\usepackage{setspace}

% Change caption format
\captionsetup[algorithm]{
    belowskip=0pt,
    aboveskip=0pt,
    % font=scriptsize,
    justification = raggedright,
    singlelinecheck = false,
    name = Algorithm
}

% Change width of horizontal lines 
\makeatletter
\newcommand{\algrule}[1][1pt]{\par\vskip.5\baselineskip\hrule height#1\vskip.5\baselineskip}
\makeatother